\chapter{はじめに}
\label{ch:introduction}

このファイルは東京理科大学創域理工学部機械航空宇宙工学科の卒業論文および同大学大学院創域理工学研究科機械航空宇宙工学専攻の修士論文を作成するにあたり,学科の論文執筆要件を満たした「非公式の」\LaTeX テンプレートです.
一連のファイルは機械航空宇宙工学専攻博士後期課程学生の松川裕樹が管理している GitHub リポジトリ\footnote{\texttt{TUS-ME\_thesis\_template}: \textless\url{https://github.com/Yuki-MATSUKAWA/TUS-ME_thesis_template}\textgreater}から入手可能です.

\section{リポジトリ内のファイル構成}
\label{sec:composition}

\begin{tcolorbox}[enhanced, title={\texttt{Yuki-MATSUKAWA/TUS-ME\_thesis\_template}}, drop fuzzy shadow]
    \begin{tabular}{ll}
        \verb|chapter/|     & 分割した \verb|tex| ファイルが入っているフォルダ \\
        \verb|figure/|      & 図が入っているフォルダ \\
        \verb|table/|       & 表の \verb|tex| ファイルが入っているフォルダ \\
        \verb|.gitignore|   & Git で管理しないファイル一覧 \\
        \verb|README.md|    & GitHub リポジトリの説明書 \\
        \verb|jsme.bst|     & 日本機械学会対応の \BibTeX スタイルファイル \\
        \verb|latexmkrc|    & 詳細は第\ref{ssec:latexmk}節を参照 \\
        \verb|main.pdf|     & \verb|main.tex| をコンパイルした \verb|pdf| ファイル \\
        \verb|main.tex|     & メインの文書ファイル \\
        \verb|mybib_en.bib| & 英語の参考文献リストファイル \\
        \verb|mybib_jp.bib| & 日本語の参考文献リストファイル \\
        \verb|settings.sty| & \verb|main.tex| で読み込むスタイルファイル
    \end{tabular}
\end{tcolorbox}



\section{このファイルの使い方}
\label{sec:howtouse}

具体的なコンパイルの方法等については第\ref{sec:makepdf}節を参照してください.


\section{卒論・修論要旨}
\label{sec:abstract}

卒業論文・修士論文を提出する際は同時に要旨が必要です.
要旨についても \LaTeX テンプレートを作成したので,GitHub リポジトリ\footnote{\texttt{TUS-ME\_thesis\_abstract}: \textless\url{https://github.com/Yuki-MATSUKAWA/TUS-ME_thesis_abstract}\textgreater}からダウンロードまたはクローンしてください.
要旨に関する詳細な説明はここでは省略します.

