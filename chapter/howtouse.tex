\chapter{環境構築・操作方法}
\label{ch:howtouse}

第\ref{ch:howtouse}章では \TeX/\LaTeX 環境構築の方法と \verb|pdf| ファイルの生成までのプロセスを説明します.
第\ref{sec:environment}節では TeX Live のインストール方法について,第\ref{sec:editor}節では \TeX/\LaTeX 対応のテキストエディター,特に VS Code の場合について述べ,第\ref{sec:makepdf}節では \verb|pdf| ファイル生成までに必要なコマンドや \verb|latexmk| の使い方,クラウド上での \LaTeX の使用について述べます.

\section{環境構築}
\label{sec:environment}

\TeX/\LaTeX を使用する際は TeX Live というディストリビューションを ご自身の PC に入れましょう.
ウイルスバスターなどのウイルス対策ソフトが TeX Live のインストールを阻害するという問題が報告されているようです.
必ず阻害するわけではありませんが,一時的に動作を停止させておいてからインストールすることをオススメします.


\subsection{Windowsの場合}
\label{ssec:windows}

ここでは ISO イメージからのインストールとネットワークインストーラーからのインストールの二種類のインストール方法を説明します.
ISO イメージからインストールの方が問題は発生しにくいかもしれません.
一方でやってみてダメならもう一方で試してみてください.

\subsubsection*{ISO イメージからインストール}

\begin{enumerate}
    \item \textless\url{https://ftp.kddilabs.jp/CTAN/systems/texlive/Images/}\textgreater から \verb|texlive.iso| をダウンロード.
    \item ダブルクリックすると BD-ROM/DVD-ROM ドライブとしてマウントされる.
    \item \verb|install-tl-windows.bat| を実行.
    \item TeX Liveインストーラが現れたら「TeXworksをインストール」のチェックを外してからインストール(もしTeXworksが欲しかったらインストールしてもよい).インストールは数時間かかることがあるので注意.
    \item インストールできたかどうかチェック.
    \begin{enumerate}
        \item \verb|Win|$+$\verb|R| でファイル名を \verb|cmd| と指定し \verb|cmd.exe|(コマンドプロンプトとも呼ぶ)を開く.
        \item \verb|tex -v| と入力し \verb|Enter|.
        \item バージョン情報が出てきたらインストール完了,出なかったら一度 Path を通してみる.
    \end{enumerate}
    \item 環境変数 Path の確認.
    \begin{enumerate}
        \item \verb|cmd.exe| を開く.\label{enum:path}
        \begin{enumerate}
            \item \verb|path| と入力し \verb|Enter|.
            \item \verb|C:\texlive\****\bin|(\verb|****| には TeX Live のバージョンにあてはまる年が入る)があれば完了.無ければ\ref{enum:system}へ.
        \end{enumerate}
        \item \verb|Win|$+$\verb|R| でファイル名を \verb|control| と入力し「コントロールパネル」を開く.\label{enum:system}
        \begin{enumerate}
            \item 「システム」$\to$「システムの詳細設定」$\to$「環境変数」の順に開く.
            \item 「システム環境変数」の「Path」をダブルクリック.
            \item \verb|C:\texlive\****\bin| があれば完了.無ければ「新規」で追加し,\ref{enum:path}へ.
        \end{enumerate}
    \end{enumerate}
\end{enumerate}


\subsubsection*{ネットワークインストーラーからのインストール}


\subsection{macOSの場合}
\label{ssec:mac}



\section{使用するエディター}
\label{sec:editor}

\TeX/\LaTeX に対応しているテキストエディターは数多く存在しますが,ここでは Microsoft が開発している Visual Studio Code(VS Code)を紹介します.
開発元は Microsoft ですが,Windows だけでなく macOS や Linux でも使用可能です.
また,VS Code には豊富な拡張機能が存在しているほか,Git との連携も非常に簡単なため近年非常に人気の高いエディターです.

\section{\texttt{pdf}ファイルの生成}
\label{sec:makepdf}

\subsection{ターミナル上での操作}
\label{ssec:terminal}

\begin{tcolorbox}[enhanced, title=\pLaTeX, drop fuzzy shadow]
\begin{verbatim}
$ platex main (複数回)
$ dvipdfmx main
\end{verbatim}
\end{tcolorbox}


\begin{tcolorbox}[enhanced, title=\upLaTeX, drop fuzzy shadow]
\begin{verbatim}
$ uplatex main (複数回)
$ dvipdfmx main
\end{verbatim}
\end{tcolorbox}


\begin{tcolorbox}[enhanced, title=\pLaTeX$+$\pBibTeX, drop fuzzy shadow]
\begin{verbatim}
$ platex main
$ pbibtex main
$ platex main (複数回)
$ dvipdfmx main
\end{verbatim}
\end{tcolorbox}


\begin{tcolorbox}[enhanced, title=\upLaTeX$+$\upBibTeX, drop fuzzy shadow]
\begin{verbatim}
$ uplatex JSME-template1
$ upbibtex JSME-template1
$ uplatex JSME-template1 (複数回)
$ dvipdfmx JSME-template1
\end{verbatim}
\end{tcolorbox}

\begin{tcolorbox}[enhanced, title=\LuaLaTeX$+$\upBibTeX, drop fuzzy shadow]
\begin{verbatim}
$ lualatex JSME-template1
$ upbibtex JSME-template1
$ lualatex JSME-template1 (複数回)
\end{verbatim}
\end{tcolorbox}


\subsection{\texttt{latexmk}を使う方法}
\label{ssec:latexmk}

\lipsum[1-8]

\subsection{クラウド上で使う方法}
\label{ssec:cloud}


\lipsum[1-4]
