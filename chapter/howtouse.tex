\chapter{環境構築・操作方法}
\label{ch:howtouse}


\section{環境構築}
\label{sec:environment}

\subsection{Windowsの場合}
\label{ssec:windows}


\subsection{macOSの場合}
\label{ssec:mac}



\section{使用するエディター}
\label{sec:editor}

\TeX, \LaTeX に対応しているテキストエディターは数多く存在存在しますが,ここでは Microsoft が開発している Visual Studio Code(VS Code)を紹介します.
開発元は Microsoft ですが,Windows だけでなく macOS や Linux でも使用可能です.
また,VS Code には豊富な拡張機能が存在しているほか,Git との連携も非常に簡単なため近年非常に人気の高いエディターです.

\section{\texttt{pdf}ファイルの生成}
\label{sec:makepdf}

\subsection{ターミナル上での操作}
\label{ssec:terminal}

\begin{tcolorbox}[enhanced, title=\pLaTeX, drop fuzzy shadow]
\begin{verbatim}
$ platex main (複数回)
$ dvipdfmx main
\end{verbatim}
\end{tcolorbox}


\begin{tcolorbox}[enhanced, title=\upLaTeX, drop fuzzy shadow]
\begin{verbatim}
$ uplatex main (複数回)
$ dvipdfmx main
\end{verbatim}
\end{tcolorbox}


\begin{tcolorbox}[enhanced, title=\pLaTeX$+$\pBibTeX, drop fuzzy shadow]
\begin{verbatim}
$ platex main
$ pbibtex main
$ platex main (複数回)
$ dvipdfmx main
\end{verbatim}
\end{tcolorbox}


\begin{tcolorbox}[enhanced, title=\upLaTeX$+$\upBibTeX, drop fuzzy shadow]
\begin{verbatim}
$ uplatex JSME-template1
$ upbibtex JSME-template1
$ uplatex JSME-template1 (複数回)
$ dvipdfmx JSME-template1
\end{verbatim}
\end{tcolorbox}

\begin{tcolorbox}[enhanced, title=\LuaLaTeX$+$\upBibTeX, drop fuzzy shadow]
\begin{verbatim}
$ lualatex JSME-template1
$ upbibtex JSME-template1
$ lualatex JSME-template1 (複数回)
\end{verbatim}
\end{tcolorbox}


\subsection{\texttt{latexmk}を使う方法}
\label{ssec:latexmk}

\lipsum[1-8]

\subsection{クラウド上で使う方法}
\label{ssec:cloud}


\lipsum[1-4]
