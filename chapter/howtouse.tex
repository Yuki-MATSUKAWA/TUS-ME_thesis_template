\chapter{環境構築・操作方法}
\label{ch:howtouse}

第~\ref{ch:howtouse}章では \TeX/\LaTeX 環境構築の方法と \verb|pdf| ファイルの生成までのプロセスを説明します.
第~\ref{sec:environment}節では TeX Live のインストール方法について,第~\ref{sec:editor}節では \TeX/\LaTeX 対応のテキストエディター,特に VS Code の場合について述べ,第~\ref{sec:makepdf}節では \verb|pdf| ファイル生成までに必要なコマンドや \verb|latexmk| の使い方,クラウド上での \LaTeX の使用について述べます.

\section{環境構築}
\label{sec:environment}

\TeX/\LaTeX を使用する際は TeX Live というディストリビューションを ご自身の PC に入れましょう.
ウイルスバスターなどのウイルス対策ソフトが TeX Live のインストールを阻害するという問題が報告されているようです.
必ず阻害するわけではありませんが,一時的に動作を停止させておいてからインストールすることをオススメします.


\subsection{Windowsの場合}
\label{ssec:windows}

ここでは ISO イメージからのインストールとネットワークインストーラーからのインストールの二種類のインストール方法を説明します.
ISO イメージからインストールの方が問題は発生しにくいかもしれません.
一方でやってみてダメならもう一方で試してみてください.

\subsubsection*{ISO イメージからインストール}

\begin{enumerate}
    \item \textless\url{https://ftp.kddilabs.jp/CTAN/systems/texlive/Images/}\textgreater から \verb|texlive.iso| をダウンロード.
    \item ダブルクリックすると BD-ROM/DVD-ROM ドライブとしてマウントされる.
    \item \verb|install-tl-windows.bat| を実行.
    \item TeX Liveインストーラが現れたら「TeXworksをインストール」のチェックを外してからインストール(もしTeXworksが欲しかったらインストールしてもよい).インストールは数時間かかることがあるので注意.
    \item インストールできたかどうかチェック.
    \begin{enumerate}
        \item \verb|Win|$+$\verb|R| でファイル名を \verb|cmd| と指定し \verb|cmd.exe|(コマンドプロンプトとも呼ぶ)を開く.
        \item \verb|tex -v| と入力し \verb|Enter|.
        \item バージョン情報が出てきたらインストール完了,出なかったら一度 Path を通してみる.
    \end{enumerate}
    \item 環境変数 Path の確認.
    \begin{enumerate}
        \item \verb|cmd.exe| を開く.\label{enum:path}
        \begin{enumerate}
            \item \verb|path| と入力し \verb|Enter|.
            \item \verb|C:\texlive\****\bin|(\verb|****| には TeX Live のバージョンにあてはまる年が入る)があれば完了.無ければ\ref{enum:system}へ.
        \end{enumerate}
        \item \verb|Win|$+$\verb|R| でファイル名を \verb|control| と入力し「コントロールパネル」を開く.\label{enum:system}
        \begin{enumerate}
            \item 「システム」$\to$「システムの詳細設定」$\to$「環境変数」の順に開く.
            \item 「システム環境変数」の「Path」をダブルクリック.
            \item \verb|C:\texlive\****\bin| があれば完了.無ければ「新規」で追加し,\ref{enum:path}へ.
        \end{enumerate}
    \end{enumerate}
\end{enumerate}


\subsubsection*{ネットワークインストーラーからのインストール}


\subsection{macOSの場合}
\label{ssec:mac}



\section{使用するエディター}
\label{sec:editor}

\TeX/\LaTeX に対応しているテキストエディターは数多く存在しますが,ここでは Microsoft が開発している Visual Studio Code(VS Code)を紹介します.
開発元は Microsoft ですが,Windows だけでなく macOS や Linux でも使用可能です.
また,VS Code には豊富な拡張機能が存在しているほか,Git との連携も非常に簡単なため近年非常に人気の高いエディターです.

\section{\texttt{pdf}ファイルの生成}
\label{sec:makepdf}

ここでは実際に \verb|pdf| ファイル(このテンプレートでは \verb|main.pdf|)を生成する過程を説明します.
これまでにある程度 \LaTeX を使った経験のある方は必要な箇所だけ読めばいい(全部わかっていれば読む必要は無い)と思います.
と言っても,\LaTeX 初心者も全部を読む必要は無く,「一旦このテンプレートで学位論文を書き上げたい」ということを考えている人は第~\ref{ssec:terminal}節の「モダン \LaTeX の場合(このテンプレートはこちら)」と第~\ref{ssec:latexmk}節を読めば大丈夫です.

\subsection{ターミナル上での操作}
\label{ssec:terminal}

第~\ref{ssec:latexmk}節の \verb|latexmk| を使用すればターミナル上での操作は非常に簡単になりますが,何か問題が発生した際にデバッグをすることを考えるとターミナル上での操作も覚えておく必要があります.
実際に \verb|pdf| ファイルを生成するときは \verb|latexmk| を使用すればいいのですが,まずはどのようなプロセスで実行されているのかを把握しておきましょう.

\subsubsection*{モダン \LaTeX の場合(このテンプレートはこちら)}

この \LaTeX テンプレートは \LuaLaTeX での執筆を前提とし,参考文献は \upBibTeX で読み込むようにしています.
詳細は下記のようになります.

\begin{tcolorbox}[enhanced, title=\LuaLaTeX$+$\upBibTeX, drop fuzzy shadow]
\begin{verbatim}
$ lualatex main
$ upbibtex main
$ lualatex main (複数回)
\end{verbatim}
\end{tcolorbox}

まずは主要な \LaTeX ソースコードの \verb|main.tex| を \LuaLaTeX で読み込むために \verb|lualatex main| とターミナルに入力します.
\verb|$| は入力しないでください.
拡張子の \verb|tex| は省略可能です.
次に,参考文献を読み込むために \verb|upbibtex main| とターミナルに入力します.
\BibTeX を使わない処理をしているときはこの操作は不要です.
これだけだとまだ \LaTeX を使う大きなメリットである相互参照の機能を使えていません.
\LaTeX で相互参照を有効にするには複数回のコンパイルが必要です.
相互参照に失敗した場合やコンパイル回数が足りていない場合は参照箇所が ? や ?? のように表示されるはずです.
そのため,\upBibTeX を読み込んだ後に ? や ?? が消えるまで複数回コンパイルしましょう.
これで \verb|main.pdf| を作成できました.

\subsubsection*{レガシー \LaTeX の場合}

モダン \LaTeX とレガシー \LaTeX の最大の違いは,\verb|pdf| ファイルを直接生成できるか否かです.
\pLaTeX や \upLaTeX のようなレガシー \LaTeX は一度 \verb|dvi| ファイルという中間ファイルを生成し,その後 \verb|dvi| ファイルを \verb|pdf| 等の適切なファイル形式に変換する作業が必要です(\verb|dvipdfmx|).
これからの時代はどんどんモダン \LaTeX に置き換えられていくと思いますが,まだ対応していない学会・論文テンプレートも多く存在しているのでここで紹介しておきます.
また,\pdfLaTeX は本来レガシー \LaTeX ですが,例外的に直接 \verb|pdf| ファイルを生成でき,国際雑誌論文テンプレートではよく使用されています.
ただし,\pdfLaTeX は日本語に対応していないため,日本語を使用したい人は \LuaLaTeX を使うようにしましょう.
どうしても \pdfLaTeX で日本語を使用したい(国際雑誌論文執筆の下書き等)場合は第~\ref{ssec:pdflatex_jp}節を参照してください.

% \begin{tcolorbox}[enhanced, title=\pLaTeX, drop fuzzy shadow]
% \begin{verbatim}
% $ platex main (複数回)
% $ dvipdfmx main
% \end{verbatim}
% \end{tcolorbox}


% \begin{tcolorbox}[enhanced, title=\upLaTeX, drop fuzzy shadow]
% \begin{verbatim}
% $ uplatex main (複数回)
% $ dvipdfmx main
% \end{verbatim}
% \end{tcolorbox}


\begin{tcolorbox}[enhanced, title=\pLaTeX$+$\pBibTeX, drop fuzzy shadow]
\begin{verbatim}
$ platex main
$ pbibtex main
$ platex main (複数回)
$ dvipdfmx main
\end{verbatim}
\end{tcolorbox}


\begin{tcolorbox}[enhanced, title=\upLaTeX$+$\upBibTeX, drop fuzzy shadow]
\begin{verbatim}
$ uplatex main
$ upbibtex main
$ uplatex main (複数回)
$ dvipdfmx main
\end{verbatim}
\end{tcolorbox}

\begin{tcolorbox}[enhanced, title=\pdfLaTeX$+$\upBibTeX, drop fuzzy shadow]
\begin{verbatim}
$ uplatex main
$ upbibtex main
$ uplatex main (複数回)
$ dvipdfmx main
\end{verbatim}
\end{tcolorbox}


\subsection{\pdfLaTeX で日本語を使用する場合}
\label{ssec:pdflatex_jp}

\subsection{\texttt{latexmk}を使う方法}
\label{ssec:latexmk}

\lipsum[1-8]

\subsection{クラウド上で使う方法}
\label{ssec:cloud}


\lipsum[1-4]
