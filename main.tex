%%%%%%%%%%%%%%%%%%%%%%%%%%%%%%%%%%%%%%%%%%%%%%%%%%%%%%%%%%%%%%%%%%%%%%%
%%%
%%%                東京理科大学 創域理工学部 機械航空宇宙工学科
%%%                   【非公式】学位論文 LaTeX テンプレート
%%%
%%%      <https://github.com/Yuki-MATSUKAWA/TUS-ME_thesis_template>
%%%
%%%                                  v1.0.0 Yuki MATSUKAWA 27 Dec. 2023
%%%
%%%%%%%%%%%%%%%%%%%%%%%%%%%%%%%%%%%%%%%%%%%%%%%%%%%%%%%%%%%%%%%%%%%%%%%

%%% 文書クラスの設定 %%%
\documentclass[
    paper=a4paper,      % A4 用紙サイズ
    report,             % report 相当の文書クラス
    fleqn,              % 数式を左寄せ
    fontsize=12pt,      % 欧文サイズ 12 pt
    jafontsize=12pt,    % 和文サイズ 12 pt
    head_space=33mm,    % 天の余白(柱とノンブルがあるので 20 mm よりも広い)
    foot_space=30mm,    % 地の余白(ノンブルが下の場合があるので 15 mm よりも広い)
    gutter=25mm,        % のどの余白
    fore-edge=10mm      % 小口の余白
    ]{jlreq}            % jlreq クラスを使用

%%% 学位論文設定ファイル %%%
\usepackage{settings}

%%% 行番号の表示 %%%
% 添削時には行番号を付けるとわかりやすい
% 提出時にはコメントアウトする
% \usepackage[mathlines,pagewise]{lineno}
% \linenumbers\relax


%%% ここから上を「プリアンブル」と言います.パッケージや独自の設定,マクロはプリアンブルや settings.styに書いてください.
%%% ここから下が論文の本体です.
\begin{document}

%%%%%%%%%%%%%%%%
%%%%% 表紙 %%%%%
%%%%%%%%%%%%%%%%

% 卒業・修了「年度」を入力
% \thesis{20**年度卒業論文}   % 卒業論文はこれ
\thesis{20**年度修士論文}   % 修士論文はこれ

% 学位論文題目
% ここには学位論文のタイトルを入れます.一文字でも間違えたら受理されません.
% タイトルが長くて改行するときは \\ を入れる.
\title{【非公式】機械航空宇宙工学科 学位論文テンプレート\\ ---\LaTeX で論文を書く際に必要な最低限の情報---}

% 卒業・修了「年」を入力
\date{20**年2月}

% 卒業論文の場合はこれ
% 大学名,学部名,学科名の間にスペースは不要
% \affiliation{東京理科大学創域理工学部機械航空宇宙工学科}

% 修士論文の場合はこれ
% 大学名,研究科名,専攻名の間にスペースは不要
\affiliation{東京理科大学大学院創域理工学研究科機械航空宇宙工学専攻}

% 研究室名を入力
\laboratory{〇〇研究室}

% 著者情報
\author{%
% 学籍番号を全角 7 桁で入力
75*****
\hskip2\zw% 学籍番号と氏名の間のスペース,消さない
% 姓と名の間は全角 1 文字スペース
姓姓 名名
} % 消さない

% 表紙の出力
\makecover

%%% 目次 %%%
\pagestyle{empty}
\def\thepage{}
\tableofcontents

%%% 記号表 %%%
\signary



\lipsum[1-10]


%%%%%%%%%%%%%%%%
%%%%% 本文 %%%%%
%%%%%%%%%%%%%%%%
\clearpage
\pagestyle{normal}
\setcounter{page}{0}
\pagenumbering{arabic}
% 上のコマンドは消さないで.
% 本文はこれ以降に記載する.

% LaTeX ソースは一つの tex ファイルに書くのではなく,章ごとの tex ファイルに分割して書きましょう.
% 分割したファイルを読み込むときは \include{xxx} または \input{xxx} を使います.
\chapter{はじめに}
\label{ch:introduction}

第~\ref{ch:introduction}~章では学位論文執筆の際の注意事項として,第~\ref{sec:template}~節でこのテンプレートの概要を,第~\ref{sec:composition}~節では GitHub リポジトリ内の各ファイルの説明を,第~\ref{sec:abstract}~節では卒論・修論要旨の \LaTeX テンプレートの紹介をします.
このテンプレートを使用する方は現在の \LaTeX 習熟度によらず必ず目を通してください.

\section{テンプレート概要}
\label{sec:template}

このファイルは東京理科大学創域理工学部機械航空宇宙工学科の卒業論文および同大学大学院創域理工学研究科機械航空宇宙工学専攻の修士論文を作成するにあたり,学科の論文執筆要件を満たした「非公式の」\LaTeX テンプレートです.
一連のファイルは東京理科大学創域理工学部機械航空宇宙工学科塚原研究室\footnote{塚原研究室ウェブページ,\textless\url{https://www.rs.tus.ac.jp/~t2lab/index-j.html}\textgreater}の GitHub Organization\footnote{\texttt{TUS-ME\_thesis\_template}, \textless\url{https://github.com/tsukahara-lab/TUS-ME_thesis_template}\textgreater} から入手可能です.
塚原研究室は熱流体系の研究室ですが,所属研究室によらずこのテンプレートは使用可能です.
パブリックリポジトリなので,他研究室所属の方もご自身の PC に入れることができます.
また,使用する際に塚原研究室の許可を取る必要はありません.
ご自由にお使いください.

このテンプレートは研究室に配属されて初めて \LaTeX で文書を書くことになった学部 4 年生を対象に,環境構築から \verb|pdf| ファイルの生成,卒業論文執筆まで滞りなく行えるように作成しています.
そのため基本事項から説明をしていますが,表紙のタイトルにもある通り「必要最低限の情報」しか記載していません.
\LaTeX 入門書は既に良書がたくさんありますが,本当の初心者は知らなくてもいい情報や学位論文執筆だけを目指すうえでは不要な情報がたくさん書かれているため,困惑した読者も多いのではないかと思います.
このテンプレートには学位論文執筆をするうえで学生が欲しがるであろう情報のみを厳選し,その情報とこのテンプレートだけあれば学位論文を書き上げるくらいのことはできるようにしておきました.
そのため,\TeX/\LaTeX で一から文書を作成することを目指している方には情報が足りていないと思います.
さらに詳しい情報が欲しい人は書籍やインターネット上の情報を参考にしてください(第~\ref{ch:information}~章を参照).
また,この \verb|main.pdf| はモダンな \LaTeX である \LuaLaTeX で作成しているほか,\verb|jlreq| というドキュメントクラスや \verb|unicode-math| など最新の機能をふんだんに使用しています.
これからこのテンプレートを使い始めるという方はモダン \LaTeX を使えるようになっておきましょう.
しかし,学会の講演論文執筆の際はこれらの機能に対応していない場合もあるため,念のためレガシーな \LaTeX のコンパイル方法等についても説明をしています.
さらに,このテンプレートでは \LaTeX に関する説明はもちろんのこと,学生が論文を書くうえで躓きやすい箇所をまとめています.
特に表記に関して細かく記載しているので参考になる箇所は多いかと思います.

もしこのテンプレートに関してバグ等,使用上の問題が発生した際は GitHub の Issues にコメントしてください.
ただし,このテンプレートを使用したことで生じた問題に関して塚原研究室および研究室に所属する個人は一切の責任を負いませんのでご了承ください.

このテンプレートを使用される皆様が無事に学位論文を執筆し,卒業・修了されることを心の底から願っております.

\begin{flushright}
    \today \\
    塚原研究室 学生有志
\end{flushright}

\clearpage
\section{リポジトリ内のファイル構成}
\label{sec:composition}

\begin{tcolorbox}[enhanced, title={\texttt{tsukahara-lab/TUS-ME\_thesis\_template}}, drop fuzzy shadow]
    \begin{tabular}{ll}
        \verb|chapter/|     & 分割した \verb|tex| ファイルが入っているフォルダ \\
        \verb|figure/|      & 図が入っているフォルダ \\
        \verb|table/|       & 表の \verb|tex| ファイルが入っているフォルダ \\
        \verb|.gitignore|   & Git で管理しないファイル一覧 \\
        \verb|README.md|    & GitHub リポジトリの説明書 \\
        \verb|jsme.bst|     & 日本機械学会対応の \BibTeX スタイルファイル \\
        \verb|latexmkrc|    & 詳細は第~\ref{ssec:latexmk}~節を参照 \\
        \verb|main.pdf|     & \verb|main.tex| をコンパイルした \verb|pdf| ファイル \\
        \verb|main.tex|     & メインの文書ファイル \\
        \verb|mybib_en.bib| & 英語の参考文献リストファイル \\
        \verb|mybib_jp.bib| & 日本語の参考文献リストファイル \\
        \verb|settings.sty| & \verb|main.tex| で読み込むスタイルファイル
    \end{tabular}
\end{tcolorbox}

\verb|README.md| はこの GitHub リポジトリを開いたときに一番最初に目に入ってくる説明書です.
内容をよく読んで使用するようにしてください.

今皆さんが読んでいるこの \verb|pdf| ファイルは \verb|main.pdf| で,\verb|main.tex| を基に作成しています.
% 今 LaTeX のソースコードを読んでいる人はこれが main.tex です.
文書のレイアウト等細かい設定は全てスタイルファイル \verb|settings.sty| に書いています.
\verb|main.tex| 冒頭の \verb|\usepackage{settings}| で読み込んでいるので間違って消さないようにしてください.
\verb|main.tex| を適切なテキストエディター(第~\ref{sec:editor}~節を参照)で開いてもらうと,\verb|\include{chapter/xxx}| と書かれた文字列が複数目に入ってくると思います.
学位論文のような長い文書を一つの \verb|tex| ファイルに書き込むとわかりにくくなるので,\verb|chapter/| 以下のディレクトリに章(chapter)ごとに分割した \verb|tex| ファイルを置いておき,それを \verb|\include{}| コマンドで読み込むようにしています.
皆さんが学位論文を執筆する際にもこのように \verb|tex| ファイルを分割しておきましょう.
また,コンパイルの際には \verb|latexmk| という機能を使用し,その際 \verb|latexmkrc| が必要になります.
\verb|latexmk| でコンパイルした際のファイル出力先を \verb|latex.out/| に設定してあります.
\verb|latexmk| の使用方法も含め,具体的なコンパイルの方法等については第~\ref{ch:howtouse}~章を参照してください.

\verb|jsme.bst|, \verb|mybib_en.bib|, \verb|mybib_jp.bib| は参考文献の出力に使用するファイル群です.
具体的な使用方法は第~\ref{ch:bibtex}~章を参照してください.

最後に,\verb|.gitignore| は Git で管理しないファイルが書かれています.
Git の詳細はここでは割愛しますが,\LaTeX で学位論文を執筆する際は Git でバージョン管理するようにしましょう.
先生や先輩に添削してもらうときに前回見せたときとの差分を \verb|latexdiff-vc|(第~\ref{sec:latexdiff-vc}~節を参照)で見せることができるほか,GitHub のプライベートリポジトリに上げることでそれ自体がバックアップとなり,大変便利です.
このリポジトリで使用している \verb|.gitignore| は GitHub で \TeX/\LaTeX に対して与えられる標準の \verb|.gitignore| を使用しています.

\section{卒論・修論要旨}
\label{sec:abstract}

卒業論文・修士論文を提出する際は同時に要旨が必要です.
要旨についても \LaTeX テンプレートを作成したので,GitHub リポジトリ\footnote{\texttt{TUS-ME\_thesis\_abstract}: \textless\url{https://github.com/Yuki-MATSUKAWA/TUS-ME_thesis_abstract}\textgreater}からダウンロードしてください.
コンパイル方法はこのテンプレートと同様です.
要旨に関する詳細な説明はここでは省略しますが,\verb|README.md| にしっかりと目を通すようにしてください.



\chapter{\LaTeX の基本}
\label{ch:basic}

\lipsum[1]

\section{\LaTeX とは何か}
\label{sec:whatislatex}

\begin{equation}
    \Re_\rmin = \frac{u_\rmin h}{\nu}, \quad \Re_\rmout = \frac{u_\rmout h}{\nu}, \quad \Re = \frac{u_0 h}{\nu}
\end{equation}

\begin{align}
    \frac{\partial u_i}{\partial t} + u_j\frac{\partial u_i}{\partial x_j} &= - \frac{1}{\rho}\frac{\partial p}{\partial x_j} + \nu\frac{\partial^2 u_i}{\partial x_j \partial x_j} \\
    \pdv{u_i}{t} + u_j\pdv{u_i}{x_j} &= - \frac{1}{\rho}\pdv{p}{x_j} + \nu\pdv{u_i}{x_j}{x_j}
\end{align}




\chapter{環境構築・操作方法}
\label{ch:howtouse}

第~\ref{ch:howtouse}~章では \TeX/\LaTeX 環境構築の方法と \verb|pdf| ファイルの生成までのプロセスを説明します.
第~\ref{sec:environment}~節では TeX Live のインストール方法について,第~\ref{sec:editor}~節では \TeX/\LaTeX 対応のテキストエディター,特に VS Code の場合について述べ,第~\ref{sec:makepdf}~節では \verb|pdf| ファイル生成までに必要なコマンドや \verb|latexmk| の使い方,クラウド上での \LaTeX の使用について述べます.

\section{環境構築}
\label{sec:environment}

\TeX/\LaTeX を使用する際は TeX Live というディストリビューションを ご自身の PC に入れましょう.
ウイルスバスターなどのウイルス対策ソフトが TeX Live のインストールを阻害するという問題が報告されているようです.
必ず阻害するわけではありませんが,一時的に動作を停止させておいてからインストールすることをオススメします.
また,この章では負荷低減のためミラーサイトからのインストール方法を説明します.

\subsection{Windowsの場合}
\label{ssec:windows}

ここでは ISO イメージからのインストールとネットワークインストーラーからのインストールの二種類のインストール方法を説明します.
ISO イメージからインストールの方が問題は発生しにくいかもしれません.
一方でやってみてダメならもう一方で試してみてください.
また, \verb|C:\Users\姓姓 名名| のように,インストールする PC のユーザー名に全角文字や空白などが入るとトラブルの原因となります.
ユーザー名を半角のものに変えてからインストールすることをおすすめします.

\subsubsection*{ISO イメージからインストール}

\begin{enumerate}
    \item \href{http://mirror.ctan.org/systems/texlive/Images/}{ミラーサイト}から \verb|texlive.iso| をダウンロード.
    \item ダブルクリックすると BD-ROM/DVD-ROM ドライブとしてマウントされる(「セキュリティの警告」が出た場合は「開く」を選択$\to$エクスプローラーで開く).
    \item 共通事項~\ref{enum:bat}へ.
\end{enumerate}


\subsubsection*{ネットワークインストーラーからのインストール}

\begin{enumerate}
    \item \href{http://mirror.ctan.org/systems/texlive/tlnet/}{ミラーサイト}から \verb|install-tl.zip| をダウンロード.
    \item \verb|install-tl.zip| を展開.
    \item 共通事項~\ref{enum:bat}へ.
\end{enumerate}

\subsubsection*{共通事項}

\begin{enumerate}
    \setcounter{enumi}{3}
    \item \verb|install-tl-windows.bat| を実行(青い警告ウィンドウが出たら「詳細情報」$\to$「実行」).\label{enum:bat}
    \item TeX Liveインストーラが現れたら「TeXworksをインストール」のチェックを外してからインストール(もしTeXworksが欲しかったらインストールしてもよい).インストールは数時間かかることがあるので注意.
    \item インストールできたかどうかチェック.
    \begin{enumerate}
        \item \verb|Win|$+$\verb|R| でファイル名を \verb|cmd| と指定し \verb|cmd.exe|(コマンドプロンプトとも呼ぶ)を開く.
        \item \verb|tex -v| と入力し \verb|Enter|.
        \item バージョン情報が出てきたらインストール完了,出なかったら一度 Path を通してみる.
    \end{enumerate}
    \item 環境変数 Path の確認.
    \begin{enumerate}
        \item \verb|cmd.exe| を開く.\label{enum:path}
        \begin{enumerate}
            \item \verb|path| と入力し \verb|Enter|.
            \item \verb|C:\texlive\****\bin|(\verb|****| には TeX Live のバージョンにあてはまる年が入る)があれば完了.無ければ\ref{enum:system}へ.
        \end{enumerate}
        \item Windows の「設定」パネルを開く.\label{enum:system}
        \begin{enumerate}
            \item 「システム」$\to$「バージョン情報」$\to$「システムの詳細設定」$\to$「環境変数」の順に開く.
            \item 「システム環境変数」の「Path」をダブルクリック.
            \item \verb|C:\texlive\****\bin| があれば完了.無ければ「新規」で追加し,\ref{enum:path}へ.
        \end{enumerate}
    \end{enumerate}
\end{enumerate}

\subsection{macOSの場合}
\label{ssec:mac}

macOS の場合は Homebrew でのインストールが簡単です.
\begin{verbatim}
    $ brew install --cask mactex-no-gui
    $ sudo tlmgr update --self --all
    $ sudo tlmgr paper a4
\end{verbatim}

\section{使用するエディター}
\label{sec:editor}

\TeX/\LaTeX に対応しているテキストエディターは数多く存在しますが,ここでは Microsoft が開発している Visual Studio Code(VS Code)を紹介します.
開発元は Microsoft ですが,Windows だけでなく macOS や Linux でも使用可能です.
また,VS Code には豊富な拡張機能が存在しているほか,Git との連携も非常に簡単なため近年非常に人気の高いエディターです.
VS Code の詳細な使用方法はここでは割愛しますが,最低限の拡張機能として \href{https://marketplace.visualstudio.com/items?itemName=James-Yu.latex-workshop}{LaTeX Workshop} を入れておくとよいでしょう.
取り扱う画像ファイルが多くなってきた場合は \href{https://marketplace.visualstudio.com/items?itemName=kisstkondoros.vscode-gutter-preview}{Image preview} があると便利です.
また,VS Code の設定ファイル \verb|settings.json|\footnote{\texttt{settings.json} の一例 \textless\url{https://gist.github.com/Yuki-MATSUKAWA/465ecd0ebcbd157e48ac1e3619c9a08c}\textgreater を紹介しておきます.\LaTeX 以外の設定も含まれているので設定の取捨選択は読者の皆さんにお任せします.} でさまざまな設定を書き加えることができます.

\section{\texttt{pdf}ファイルの生成}
\label{sec:makepdf}

ここでは実際に \verb|pdf| ファイル(このテンプレートでは \verb|main.pdf|)を生成する過程を説明します.
これまでにある程度 \LaTeX を使った経験のある方は必要な箇所だけ読めばいい(全部わかっていれば読む必要は無い)と思います.
と言っても,\LaTeX 初心者も全部を読む必要は無く,\textcolor{red}{「一旦このテンプレートで学位論文を書き上げたい」ということを考えている人は第~\ref{ssec:terminal}~節の「\LuaLaTeX の場合(このテンプレートはこちら)」と第~\ref{ssec:latexmk}~節を読めば大丈夫}です.

\subsection{ターミナル上での操作}
\label{ssec:terminal}

第~\ref{ssec:latexmk}~節の \verb|latexmk| を使用すればターミナル上での操作は非常に簡単になりますが,何か問題が発生した際にデバッグをすることを考えるとターミナル上での操作も覚えておく必要があります.
実際に \verb|pdf| ファイルを生成するときは \verb|latexmk| を使用すればいいのですが,まずはどのようなプロセスで実行されているのかを把握しておきましょう.

\subsubsection*{\LuaLaTeX の場合(このテンプレートはこちら)}

この \LaTeX テンプレートは \LuaLaTeX での執筆を前提とし,参考文献は \upBibTeX で読み込むようにしています.
\LuaLaTeX は速度がやや遅いものの,高機能で Unicode に対応しているため近年人気が出てきているモダンな \LaTeX です.
使い方の詳細は下記のようになります.

\begin{tcolorbox}[enhanced, title=\LuaLaTeX$+$\upBibTeX, drop fuzzy shadow]
\begin{verbatim}
$ lualatex main
$ upbibtex main
$ lualatex main (複数回)
\end{verbatim}
\end{tcolorbox}

まずは主要な \LaTeX ソースコードの \verb|main.tex| を \LuaLaTeX で読み込むために \verb|lualatex main| とターミナルに入力します.
\verb|$| は入力しないでください.
拡張子の \verb|.tex| は省略可能です.
次に,参考文献を読み込むために \verb|upbibtex main| とターミナルに入力します.
\BibTeX を使わない処理をしているときはこの操作は不要です.
これだけだとまだ \LaTeX を使う大きなメリットである相互参照の機能を使えていません.
\LaTeX で相互参照を有効にするには複数回のコンパイルが必要です.
相互参照に失敗した場合やコンパイル回数が足りていない場合は参照箇所が ? や ?? のように表示されるはずです.
そのため,\upBibTeX を読み込んだ後に ? や ?? が消えるまで複数回コンパイルしましょう.
これで \verb|main.pdf| を作成できました.

\subsubsection*{レガシー \LaTeX の場合}

モダン \LaTeX とレガシー \LaTeX の最大の違いは,\verb|pdf| ファイルを直接生成できるか否かです.
\pLaTeX や \upLaTeX のようなレガシー \LaTeX は一度 \verb|dvi| ファイルという中間ファイルを生成し,その後 \verb|dvi| ファイルを \verb|pdf| 等の適切なファイル形式に変換する作業が必要です(\verb|dvipdfmx|).
これからの時代はどんどんモダン \LaTeX に置き換えられていくと思いますが,まだ対応していない学会・論文テンプレートも多く存在しているのでここで紹介しておきます.
また,\pdfLaTeX は本来レガシー \LaTeX ですが,例外的に直接 \verb|pdf| ファイルを生成でき,国際雑誌論文テンプレートではよく使用されています.
ただし,\pdfLaTeX は日本語に対応していないため,日本語を使用したい人は \LuaLaTeX を使うようにしましょう.
どうしても \pdfLaTeX で日本語を使用したい(国際雑誌論文執筆の下書き等)場合は第~\ref{ssec:pdflatex_jp}~節を参照してください.

\pLaTeX は日本語に対応した \LaTeX として長年愛用されてきましたが,今は \LuaLaTeX などに置き換えられてきています.
皆さんは使わないようにしましょう.
使い方は下記の通り.
\LuaLaTeX の項目と同様,\BibTeX を使わない場合はそこのコマンドを省略してください.

\begin{tcolorbox}[enhanced, title=\pLaTeX$+$\pBibTeX, drop fuzzy shadow]
\begin{verbatim}
$ platex main
$ pbibtex main
$ platex main (複数回)
$ dvipdfmx main
\end{verbatim}
または
\begin{verbatim}
$ ptex2pdf -l main
$ pbibtex main
$ ptex2pdf -l main (複数回)
\end{verbatim}
\end{tcolorbox}

上記コマンドの \verb|ptex2pdf -l main| は \verb|platex main| と \verb|dvipdfmx main| を続けて実行するコマンドです.

次に \upLaTeX について説明します.
これは \pLaTeX を Unicode に対応させたものとなっており,現在でも広く使われています.
そのため,このテンプレートを使用することだけを考える際は不要な情報ですが,念のため載せておきます.
\upLaTeX では \upBibTeX が使えますが先程と同様,不要な場合は省略してください.

\begin{tcolorbox}[enhanced, title=\upLaTeX$+$\upBibTeX, drop fuzzy shadow]
\begin{verbatim}
$ uplatex main
$ upbibtex main
$ uplatex main (複数回)
$ dvipdfmx main
\end{verbatim}
または
\begin{verbatim}
$ ptex2pdf -l -u main
$ pbibtex main
$ ptex2pdf -l -u main (複数回)
\end{verbatim}
\end{tcolorbox}

\pLaTeX はレガシー \LaTeX ですが例外的に直接 \verb|pdf| を出力できます(\verb|dvipdfmx| が不要).
日本語には対応していませんが,国際雑誌論文では広く使用されています.
どうしても \pdfLaTeX で日本語を使用したい場合は次の第~\ref{ssec:pdflatex_jp}~節を参照.
使い方は下記の通り.

\begin{tcolorbox}[enhanced, title=\pdfLaTeX$+$\upBibTeX, drop fuzzy shadow]
\begin{verbatim}
$ pdflatex main
$ upbibtex main
$ pdflatex main (複数回)
\end{verbatim}
\end{tcolorbox}


\subsection{\pdfLaTeX で日本語を使用する場合}
\label{ssec:pdflatex_jp}

国際雑誌論文等のコンパイルは \pdfLaTeX が想定されていることがあります.
\pdfLaTeX はレガシー \LaTeX でありながらも直接 \verb|pdf| ファイルを生成できることから海外では広く使用されていますが,残念ながら日本語に対応していません.
しかし,英語論文の下書きとして日本語を使いたい場合があると思います.
その際に,見た目が少し悪くなるものの \pdfLaTeX で日本語を使用する方法が一応あるのでここで紹介しておきます.

\begin{tcolorbox}[enhanced, title=文書全体で日本語を使用, drop fuzzy shadow]
\begin{verbatim}
\usepackage[whole]{bxcjkjatype}
\end{verbatim}
\end{tcolorbox}

まず,\LaTeX 文書全体で日本語を使用したい場合は上記のコマンドをプリアンブルに書きます.
これで文書全体で日本語の使用が可能になります.
ただし,前述の通り見た目が悪くなるので下書き用(後で英語に変更する用)として使用してください.

\begin{tcolorbox}[enhanced, title=文書の一部分で日本語を使用, drop fuzzy shadow]
\begin{verbatim}
プリアンブルに記載
\usepackage{CJKutf8}

本文中に記載
\begin{CJK}{UTF8}{ipxm}
日本語
\end{CJK}
\end{verbatim}
\end{tcolorbox}

次に,文書全体ではなく一部分でのみ日本語を使用したい場合のコマンドは上記のようになっています.
まず,\verb|\usepackage{CJKutf8}| というパッケージを読み込むことで日本語を使用できるようにします.
厳密には日本語だけでなく,中国語(\textbf{C}hinese),日本語(\textbf{J}apanese),韓国語(\textbf{K}orean)の組版規則に対応させるためのパッケージとなります.
次に本文中の日本語を使いたい箇所を \verb|\begin{CJK}{UTF8}{ipxm}| と \verb|\end{CJK}| で囲ってあげればそこでは日本語を使えるようになります.
米国物理学協会(American Institute of Physics, AIP)が発行している雑誌論文(Physics of Fluids など)は著者の氏名で英語表記以外に漢字等の表記を併記することが可能になっています.
このようなときにこのコマンドを使ってあげるとよいでしょう.
また,日本語を使う箇所がもう少し長い場合はプリアンブルで \verb|\newcommand*{\Ja}[1]{\begin{CJK}{UTF8}{ipxm}#1\end{CJK}}| のようにコマンドを作ってあげてもいいかもしれません.

\subsection{\texttt{latexmk}を使う方法}
\label{ssec:latexmk}

\LaTeX 関連のファイルが変更されるたびに第~\ref{ssec:terminal}~節で紹介した操作を毎回行うのは非常に面倒です.
そこで \verb|latexmk| という機能を使って簡略化しましょう.
\verb|latexmk| を使うと,このリポジトリ内に入っている \verb|latexmkrc| というファイル\footnote{拡張子はつけないでください.}を呼び出し,実行したいコマンドを一回の操作で実行してくれます.
\verb|latexmkrc| は \verb|main.tex|(主要な \LaTeX コード)と同じ階層に用意しておいてください.
あとはターミナル上で \verb|latexmk main| と打てばすべて実行してくれます.

\begin{tcolorbox}[enhanced, title=\texttt{latexmk}を使用, drop fuzzy shadow]
\begin{verbatim}
$ latexmk main
\end{verbatim}
\end{tcolorbox}

これで随分楽になったと思いますが,VS Code を使っている皆さんはもっと楽にできます.
私が使っている \verb|settings.json| の中で \LaTeX のビルド時に \verb|latexmk| で実行するように設定してあるので,Windows の場合は \verb|Ctrl|+\verb|Alt|+\verb|B| で同様の操作を行ってくれます.
\verb|settings.json| の設定を変更して自動コンパイルにすることもできます.
また,Windows で \verb|pdf| ファイルのプレビューを見たい場合は \verb|Ctrl|+\verb|Alt|+\verb|V| の操作で表示できます.
\verb|Ctrl| キーを押しながらマウスでプレビューをクリックすると該当箇所のソースコードに飛べるのも便利な機能です.

\subsection{クラウド上で使う方法}
\label{ssec:cloud}

第~\ref{sec:environment}~節で環境構築の方法を述べました.
本音としては \LaTeX の全ユーザーが自身の PC にローカルの \LaTeX 環境を整えてほしいのですが,環境構築に手間がかかる不便さもあるため,ここでは TeX Live 等のインストールをせずにクラウド上での \LaTeX 環境構築方法を説明します.
クラウド上で \LaTeX を使用できるツールとしては Cloud LaTeX や Overleaf といったものが有名で,私は Overleaf をよく使っているのでここでは Overleaf の説明をします.

Overleaf は複数のユーザーによる(同時)共同執筆も可能となっており,Overleaf 自体が Git と同様の役割を担っているため大変便利です.
もちろん Git/GitHub との連携も可能となっています.

\begin{tcolorbox}[enhanced, drop fuzzy shadow]
    後で書きます.
\end{tcolorbox}




\chapter{\BibTeX による参考文献一覧の出力}
\label{ch:bibtex}

この章では参考文献の一般的な記載方法,\LaTeX での処理方法について説明します.
第~\ref{sec:bibcaution} 節はこのテンプレートに限らない,一般的に学術論文等で参考文献を載せる際に注意すべき点をまとめているので読み飛ばしていただいても構いません.

\section{参考文献の記載時の一般的な注意事項}
\label{sec:bibcaution}

論文執筆の際には多くの文献を引用することになります.
読者に正しい情報を提供するのはもちろんのこと,先人たちの業績を認め,評価するという観点でも文献を引用する際は細心の注意を払いましょう.

\subsection{引用方式}
\label{ssec:citation_style}

参考文献の引用方法は Harvard 方式と Vancouver 方式に大別できます.
\begin{itemize}
    \item Harvard 方式
    \begin{itemize}
        \item 本文中での引用はいわゆる author-year 方式.「著者名」と「発行年」を記載する.
        \item 本文中での引用は苗字だけでの記載が多い.引用例:
        \begin{itemize}
            \item 著者 1 名:\cite{Reynolds:PhilTransRoySoc1883}
            \item 著者 2 名:\cite{Schmid:Springer2001}
            \item 著者 3 名以上:\cite{Berghout:JFM2020}
        \end{itemize}
        \item et al. はラテン語で「その他」を意味する et alii の略.\textit{Italic} 体で書くことも多い.
        \item 論文末尾の文献リストは著者名のアルファベット順でソート.
    \end{itemize}
    \item Vancouver 方式
    \begin{itemize}
        \item 本文中での引用は番号.
        \item 本文中での引用例:~が明らかになっている $^{[1,2]}$.
        \item 論文末尾の文献リストは本文での登場順でソート.
    \end{itemize}
\end{itemize}
日本機械学会の論文執筆規定では Harvard 方式になっているため,この学位論文テンプレートも Harvard 方式を採用しています.

\subsection{文献リストの作り方}
\label{ssec:bib_list}


\begin{tcolorbox}[enhanced, drop fuzzy shadow]
    後で書きます.
    sentence case とか.
\end{tcolorbox}


\subsection{文献の正しい引用の仕方}
\label{ssec:cite_correctly}

\begin{tcolorbox}[enhanced, drop fuzzy shadow]
    後で書きます.
    孫引きとか.
    版・刷とか.
\end{tcolorbox}


\section{\BibTeX の使い方}
\label{sec:howtouse_bibtex}

\LaTeX では文献リストを作る方法として \verb|thebibliography| 環境の中で \verb|\bibitem| コマンドを使用する方法があります.
しかし,この方法は文献リストを人間が全て手打ちで入力しなければいけません.
引用する文献が片手で数え切れるくらいの数であれば全て手打ちで文献リストを作ってもいいかもしれませんが,学術論文や学位論文になると人力で文献リストを作るのは時間の無駄ですしミスの元になります\footnote{学会の講演論文テンプレート等では,人力で文献リストを作る方法が採用されているものが結構あります.\texttt{\textbackslash bibitem} コマンド等を使って自力でリストを作成する方法は大抵の \LaTeX 入門書に記載があるのでそちらを参考にしてください.}.

そこで,\BibTeX を使用した参考文献リストの作成方法を説明します.
\BibTeX を使えばユーザーが作成した \verb|bib| ファイルを読み込んで自動で文献リストを作ってくれます\footnote{\BibTeX は現在でも広く使われていますが,最近は \BibTeX の後継として \texttt{biblatex} が徐々に普及してきています.この学位論文テンプレートでは後述の \texttt{jsme.bst} を使用しているため \BibTeX の内容に限定して記載しています.将来的には \texttt{biblatex} に置き換えたいと考えています.}.

\subsection{\texttt{jsme.bst} について}
\label{ssec:jsme-bst}

第~\ref{sec:bibcaution} 節で文献の一般的な引用の仕方を説明しましたが,学会やジャーナルによって細かいルールは異なります.
それぞれの論文での引用ルールに則った出力を得るために必要なものが \BibTeX スタイルファイル,\verb|bst| ファイルです.
\BibTeX を走らせるときは \verb|bst| ファイルを読み込んで文献リストの出力方法を決めます.
有名な \verb|bst| ファイルとして \verb|jplain.bst| や \verb|jecon.bst| があります.

東京理科大学創域理工学部機械航空宇宙工学科の卒業論文では,参考文献一覧および本文中での引用に関して一般社団法人日本機械学会の論文執筆テンプレートの書き方に沿って記載するよう決められています.
しかし,日本機械学会から公式な \BibTeX スタイルテンプレートは配布されていません.
そこで,塚原研究室所属の学生が日本機械学会の参考文献の書き方を再現した「非公式の」\BibTeX スタイルテンプレート\footnote{\texttt{JSME-bst}, \textless\url{https://github.com/Yuki-MATSUKAWA/JSME-bst}\textgreater}を開発し,GitHub で公開しているのでこれを使用します.
使用方法は一般的な \BibTeX と同様ですが,詳細な説明書(\href{https://github.com/Yuki-MATSUKAWA/JSME-bst/blob/main/JSME-template1.pdf}{\texttt{JSME-template1.pdf}})がリポジトリ内にあるので何か問題があった場合はそれを読むようにしましょう.
日本機械学会の規定通りの文献出力を得るには \verb|jsme.bst| を使用すれば大丈夫ですが,\verb|TUS-ME_thesis_template| リポジトリ内には最初から \verb|jsme.bst| が入っているのでこれを読んでいる皆さんが新たに \verb|jsme.bst| ファイルを \verb|JSME-bst| リポジトリから移してくる必要はありません.

\subsection{\texttt{bib} ファイルについて}
\label{ssec:bib-file}

\BibTeX は自動で文献リストを作ってくれるとは言ったものの,書誌情報は与えてあげないといけません.
\verb|bib| ファイルには自分が引用する書誌情報を記載します.
\verb|bib| ファイルの書き方は \verb|jsme.bst| 内の \href{https://github.com/Yuki-MATSUKAWA/JSME-bst/blob/main/JSME-template1.pdf}{\texttt{JSME-template1.pdf}} で詳細に書いてあるのでそちらをよく読んでください.
\BibTeX 初心者にとっても痒い所に手が届くように書かれています.
ただ,基本的な内容だけここにも書いておきます.

\verb|bib| ファイルに入力する書誌情報は次のような構造になっています.
\begin{tcolorbox}[enhanced, title=\textgt{\texttt{bib} ファイル内の書誌情報の構造}, drop fuzzy shadow]
\begin{verbatim}
@エントリー名{参照キー,
    フィールド1 = {},
    フィールド2 = {},
    フィールド3 = {}
}
\end{verbatim}
\end{tcolorbox}
\noindent
だいたいの雑誌論文のウェブサイトでは \BibTeX 形式で書誌情報を出力できる機能があるのでそこから \verb|bib| ファイルをダウンロードします.
もちろん,ダウンロードした \verb|bib| ファイルを自分で書き換えることもできますし,自分で一から \verb|bib| ファイルを作成することも可能です.
文献を本文中で引用する際は \verb|\citet{Matsukawa:PoF2022}| のように書きます.
このときの \verb|Matsukawa:PoF2022| が参照キーです.
参照キーの書き方に特に規則は無く,半角カンマ以外の半角記号も使用可能です.
ただ,自分の中でマイルールを設けておくと引用する際に楽です.




\chapter{先生や先輩に添削してもらうときの注意点}
\label{ch:check}


\section{\texttt{latexdiff}}
\label{sec:latexdiff}



\section{\texttt{latexdiff-vc}}
\label{sec:latexdiff-vc}



\chapter{さらに詳しい情報が欲しい人は}
\label{ch:information}


\section{書籍}
\label{sec:book}

\subsection*{初心者向け}


\subsection*{中級者以上向け}


\section{インターネット上の情報}
\label{sec:internet}





%%% 謝辞 %%%
%%%%%%%%%%%%%%%%
%%%%% 謝辞 %%%%%
%%%%%%%%%%%%%%%%
\acknowledge

謝辞は非常に重要な章です.
学位論文執筆にあたりお世話になった人は必ず書きましょう.
私は基本的に謝辞は自由に書いてほしいと思っていますが,明らかな間違いを書いてくる人を結構見てきました.
これまで添削してきた論文の中でよく見た間違いや書き方の例などをここでまとめておきます.

まずは全体に共通する注意事項です.
\begin{itemize}
    \item 論文本体は基本的に常体(だ・である)で書きますが,謝辞に限り敬体(です・ます)で書いても構いません.
    \item 氏名の書き間違い等には十分気をつけてください.
    間違えると大変失礼です.
    心配な場合は大学や研究室の web ページ等からコピーアンドペーストするとよいでしょう.
    \item 基本的な日本の大学教員の職階(役職名)は上から順に以下の通りです.
    \begin{enumerate}
        \item 教授
        \item 准教授
        \item 講師
        \item 助教
        \item 助手
    \end{enumerate}
    大学によっては上記職階の一部が設置されていない場合があります.
    また,よくある間違いとして,准教授や助教を「助教授」と書くことが挙げられます.
    助教授はかつて実際に存在した職階ですが,2007 年の学校教育法の改正に伴い,准教授へと名称が変更されました.
    また,助教が助教授の略称だと勘違いしているケースもよく見ます.
    \item 普段大学で○○教授と呼ぶことはありませんが,学位論文の謝辞では「フルネーム$+$役職名」で記載しましょう.
    ○○教授や○○准教授と比較して○○講師や○○助教という言い方はあまり聞きませんが,統一して役職名で書いてください.
    その際,役職名を間違えると大変失礼です.
    卒業論文執筆時に准教授だった先生が修士論文執筆時には教授に昇進していることもあります.
    卒論の謝辞を使いまわすのではなく,よく確認しておくように.
    \item 修士課程や博士後期課程における「課程」を「過程」と書く間違いをよく見るので気をつけましょう.
    \item また,大学の場合は「卒業」で大学院の場合は「修了」です.
    「修了」を「終了」と書いてしまうミスもたまに見ます.
    \item 謝辞はどうしても同じような文言や文末表現が連続しがちです.
    しかし,感謝を伝えるためにも表現を少しずつ変えてください.
\end{itemize}

次に,おおまかな書く順番とそれぞれの注意事項は下記の通りです.
必ずこの順番を守らなければいけないというわけではありませんが,このような順であることが多いです.
\begin{enumerate}
    \item 指導教員(主査)
    \begin{itemize}
        \item 書き方の例:指導教員である東京理科大学創域理工学部機械航空宇宙工学科の○○教授は……
        \item 大学によっては教員の所属が大学院の場合もありますが,東京理科大学の場合は学部です.
        \item よくある間違いとして「指導教官の……」と書くものです.教官という言い回しは昔の国立大学の教員のものです.
        正しくは「指導教員」です.
        \item 稀に実質的な指導教員と主査が異なる場合があります.
    \end{itemize}
    \item 共同研究者
    \begin{itemize}
        \item 書き方の例:例:○○研究所の○○博士は……
        \item 共同研究者は大学教員の場合もあれば大学以外の研究機関の職員の場合もあります.所属名と役職名をよく確認しておきましょう.
    \end{itemize}
    \item 副査の先生方
    \begin{itemize}
        \item 書き方の例:○○教授と○○准教授には本論文の副査を引き受けていただき……
        \item 卒業論文の場合は副査がありませんが,修士論文の場合は 2 名の副査があります.
    \end{itemize}
    \item その他お世話になった学生以外の先生方・研究者
    \begin{itemize}
        \item 書き方の例:本研究室所属の○○博士研究員は……
        \item 研究室に博士研究員(いわゆるポスドク)や直接の自分の指導教員ではない助教や秘書の方がいる場合も忘れずに書いておきましょう.
    \end{itemize}
    \item 研究助成元や計算機センター等
    \begin{itemize}
        \item 書き方の例:直接数値計算の一部は東北大学サイバーサイエンスセンター大規模計算システム AOBA を利用しました.
        \item 研究遂行にあたり助成等を受けた場合に記載します.奨学金をもらっている場合もここに書きましょう.
    \end{itemize}
    \item 研究室メンバー
    \begin{itemize}
        \item 書き方の例:博士後期課程 1 年の○○氏は……
        \item 研究室メンバー全員を列挙しなければいけないわけではありません.論文執筆にあたって本当に貢献した人を書いてください.全員の貢献があっての学位論文だと思うのであれば全員書いてもいいと思います.
        \item 並べ方は基本的に貢献度順です.自分と同じ研究班の人をまずは優先的に書きましょう.
        \item 貢献度が同じだと判断した場合は上からの学年順に並べましょう.
        \item 貢献度と学年が同じ人は五十音順に並べましょう.
        \item 日本学術振興会特別研究員に採用されている博士後期課程学生がいる場合は肩書きとして併記することが多いです.
    \end{itemize}
    \item 家族
    \begin{itemize}
        \item 書き方の例:最後に,私の大学院進学に対して理解を示し,常に私を気にかけていただいた祖母と両親に感謝申し上げます.
        \item 自分を応援してくれた家族への感謝も忘れずに.
        \item 残念ながら自分の卒業・修了を見届けられずにご家族が亡くなってしまう場合もあります.その際は「天国で見守ってくれている○○」などと書くことが多いです.
        \item なお,家庭環境等の事情により家族を記載したくない場合はこの限りではありません.心の中で感謝しておきましょう.
    \end{itemize}
\end{enumerate}



%%% 文献 %%%
\biblist
% 使用する bst ファイル
\bibliographystyle{jsme}
% 読み込む bib ファイル
\bibliography{
    mybib_en.bib,
    mybib_jp.bib
}

%%% 付録 %%%
%%%%%%%%%%%%%%%%
%%%%% 付録 %%%%%
%%%%%%%%%%%%%%%%
\appendix
\label{ch:app}
\pagestyle{appendix}

\chapter{修士課程における研究成果}
\label{ch:app_master}

\lipsum[1-8]

\chapter{スーパーコンピューターごとの性能比較}
\label{ch:app_sx}

\lipsum[1-2]

\section{スパコンXXX}
\label{sec:app_xxx}

\lipsum[1-4]

\section{スパコンYYY}
\label{sec:app_yyy}

\lipsum[1-4]


%%%%%%%%%%%%%%%%%%%%%%%%%%%%%%
%%% 文章を書けるのはここまで %%%
%%%%%%%%%%%%%%%%%%%%%%%%%%%%%%
\end{document}
