%%%%%%%%%%%%%%%%%%%%%%%%%%%%%%%%%%%%%%%%%%%%%%%%%%%%%%%%%%%%%%%%%%%%%%%
%%%
%%%                東京理科大学 創域理工学部 機械航空宇宙工学科
%%%                   【非公式】学位論文 LaTeX テンプレート
%%%
%%%      <https://github.com/Yuki-MATSUKAWA/TUS-ME_thesis_template>
%%%
%%%                                  v1.0.0 Yuki MATSUKAWA 27 Dec. 2023
%%%
%%%%%%%%%%%%%%%%%%%%%%%%%%%%%%%%%%%%%%%%%%%%%%%%%%%%%%%%%%%%%%%%%%%%%%%

% \RequirePackage{fancyhdr}
% A4 用紙サイズ,欧文・和文サイズ 12 pt
\documentclass[paper=a4paper,report,fontsize=12pt,jafontsize=12pt]{jlreq}

% 学位論文設定ファイル
\usepackage{settings}
% ダミーテキスト生成パッケージ
\usepackage{lipsum}

% \ModifyPageStyle{headings}{
%     running_head_position=top-left,
%     nombre_position=top-right,
%     mark_format={_chapter={第 \thechapter 章 \quad #1}},
%     odd_running_head={_chapter},even_running_head={_chapter}}
    % odd_running_head=\underline{\bfseries 第\thechapter 章 \quad #1}}
% \NewPageStyle{normal}{running_head_position=top-left}
% \ModifyPageStyle{myheadings}{
%     \def\@evenhead{\hbox to\textwidth{\thepage\hfil{\bfseries \leftmark}}\llap{\rule[-.5zw]{\textwidth}{.6pt}}}
%     \def\@oddhead{{\bfseries \rightmark}\hfil\thepage\llap{\rule[-.5zw]{\textwidth}{.6pt}}}%
% }

% \pagestyle{myheadings}

%%% ヘッダー・フッター
% \usepackage{fancyhdr}

\begin{document}

%%%%%%%%%%%%%%%%
%%%%% 表紙 %%%%%
%%%%%%%%%%%%%%%%

% 卒業・修了「年度」を入力
\thesis{20**年度卒業論文}

\title{ここには学位論文のタイトルを入れます.\\ 一文字でも間違えたら受理されません.}

% 卒業・修了「年」を入力
\date{20**年2月}
% 卒業論文の場合はこれ
\affiliation{東京理科大学創域理工学部機械航空宇宙工学科}
% 修士論文の場合はこれ
% \affiliation{東京理科大学大学院創域理工学研究科機械航空宇宙工学専攻}
% 研究室名を入力
\laboratory{〇〇研究室}
\author{
% 学籍番号を全角 7 桁で入力
75*****
\hskip2\zw % 学籍番号と氏名の間のスペース,消さない
% 姓と名の間は全角 1 文字スペース
姓姓 名名
} % 消さない

\makecover

\newpage
\setcounter{page}{0}
\pagenumbering{arabic}
% \pagestyle{headings}

\chapter{序論}
\label{ch:introduction}

\lipsum[1]

\section{研究背景}
\label{sec:background}

\lipsum[1-8]

\chapter{数値計算手法}
\label{ch:method}

\lipsum[1]

\section{対象}
\label{sec:target}

\lipsum[1-8]


% \clearpage
% %%% ハイパーリンクのズレを調整
% \phantomsection
% %%% 参考文献内の URL 表示をタイプライター調にしない
% \renewcommand\UrlFont{\rmfamily}
% %%% \nocite{*}が有効のとき,引用していない文献も含めて全て表示
% \nocite{*}
% %%% 目次に「文献」を追加
% \addcontentsline{toc}{section}{\refname}
% %%% 使用する bst ファイル
% \bibliographystyle{jsme}
% %%% 読み込む bib ファイル
% \bibliography{
% mybib_en.bib,
% mybib_jp.bib
% }

\end{document}

