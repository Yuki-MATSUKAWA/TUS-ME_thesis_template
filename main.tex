%%%%%%%%%%%%%%%%%%%%%%%%%%%%%%%%%%%%%%%%%%%%%%%%%%%%%%%%%%%%%%%%%%%%%%%
%%%
%%%                東京理科大学 創域理工学部 機械航空宇宙工学科
%%%                   【非公式】学位論文 LaTeX テンプレート
%%%
%%%      <https://github.com/Yuki-MATSUKAWA/TUS-ME_thesis_template>
%%%
%%%                                  v1.0.0 Yuki MATSUKAWA 27 Dec. 2023
%%%
%%%%%%%%%%%%%%%%%%%%%%%%%%%%%%%%%%%%%%%%%%%%%%%%%%%%%%%%%%%%%%%%%%%%%%%

% A4 用紙サイズ,欧文・和文サイズ 12 pt
\documentclass[
    openany,oneside,
    paper=a4paper,
    book,
    fontsize=12pt,
    jafontsize=12pt,
    head_space=30mm]{jlreq}
% \documentclass[paper=a4paper,report,fontsize=12pt,jafontsize=12pt]{jlreq}

% 学位論文設定ファイル
\usepackage{settings}
\usepackage{layout}

%%% ここから上を「プリアンブル」と言います
%%% ここから下が論文の本体です
\begin{document}
\frontmatter
% \layout
%%%%%%%%%%%%%%%%
%%%%% 表紙 %%%%%
%%%%%%%%%%%%%%%%
% 卒業・修了「年度」を入力
\thesis{20**年度卒業論文}
% 学位論文題目
\title{ここには学位論文のタイトルを入れます.\\ 一文字でも間違えたら受理されません.}
% 卒業・修了「年」を入力
\date{20**年2月}
% 卒業論文の場合はこれ
\affiliation{東京理科大学創域理工学部機械航空宇宙工学科}
% 修士論文の場合はこれ
% \affiliation{東京理科大学大学院創域理工学研究科機械航空宇宙工学専攻}
% 研究室名を入力
\laboratory{〇〇研究室}
% 著者情報
\author{%
% 学籍番号を全角 7 桁で入力
75*****
\hskip2\zw% 学籍番号と氏名の間のスペース,消さない
% 姓と名の間は全角 1 文字スペース
姓姓 名名
} % 消さない

\makecover

%%%%%%%%%%%%%%%%
%%%%% 目次 %%%%%
%%%%%%%%%%%%%%%%
\pagestyle{empty}
\def\thepage{}
\tableofcontents

%%%%%%%%%%%%%%%%
%%%% 記号表 %%%%
%%%%%%%%%%%%%%%%
\signary

\lipsum[1-8]


%%%%%%%%%%%%%%%%
%%%%% 本文 %%%%%
%%%%%%%%%%%%%%%%
\mainmatter
\newpage
\setcounter{page}{0}
\pagenumbering{arabic}

\chapter{序論}
\label{ch:introduction}

\lipsum[1]

\section{研究背景}
\label{sec:background}

\lipsum[1-8]

\section{先行研究}
\label{sec:previous}

\lipsum[1-2]

\subsection{〇〇の先行研究}
\label{ssec:marumaru}

\lipsum[1-4]

\subsection{××の先行研究}
\label{ssec:batsubatsu}

\lipsum[1-4]

\chapter{数値計算手法}
\label{ch:method}

\lipsum[1]

\section{対象}
\label{sec:target}

\lipsum[1-8]


\chapter{hoge}

\chapter{hoge}

\chapter{hoge}

\chapter{hoge}

\chapter{hoge}

\chapter{hoge}

\chapter{hoge}

\chapter{hoge}

\chapter{hoge}

\chapter{hoge}

\chapter{hoge}

\chapter{hoge}

\chapter{hoge}

\chapter{hoge}

\chapter{hoge}

\chapter{hoge}

\chapter{hoge}

\chapter{hoge}

\chapter{hoge}

\chapter{hoge}

\chapter{hoge}

\chapter{hoge}

\chapter{hoge}





%%%%%%%%%%%%%%%%
%%%%% 謝辞 %%%%%
%%%%%%%%%%%%%%%%
\acknowledge

\lipsum[1-8]


%%%%%%%%%%%%%%%%
%%%%% 文献 %%%%%
%%%%%%%%%%%%%%%%
% \clearpage
% %%% ハイパーリンクのズレを調整
% \phantomsection
% %%% 参考文献内の URL 表示をタイプライター調にしない
% \renewcommand\UrlFont{\rmfamily}
% %%% \nocite{*}が有効のとき,引用していない文献も含めて全て表示
% \nocite{*}
% %%% 目次に「文献」を追加
% \addcontentsline{toc}{section}{\refname}
% %%% 使用する bst ファイル
% \bibliographystyle{jsme}
% %%% 読み込む bib ファイル
% \bibliography{
% mybib_en.bib,
% mybib_jp.bib
% }

%%%%%%%%%%%%%%%%
%%%%% 付録 %%%%%
%%%%%%%%%%%%%%%%
\appendix
\label{ch:app}
\pagestyle{appendix}

\chapter{修士課程における研究成果}
\label{ch:app_master}

\lipsum[1-8]

\chapter{スーパーコンピューターごとの性能比較}
\label{ch:app_sx}

\lipsum[1-2]

\section{スパコンXXX}
\label{sec:app_xxx}

\lipsum[1-4]

\section{スパコンYYY}
\label{sec:app_yyy}

\lipsum[1-4]


%%% 文章を書けるのはここまで
\end{document}
