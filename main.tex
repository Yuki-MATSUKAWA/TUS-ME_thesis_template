%%%%%%%%%%%%%%%%%%%%%%%%%%%%%%%%%%%%%%%%%%%%%%%%%%%%%%%%%%%%%%%%%%%%%%%
%%%
%%%                東京理科大学 創域理工学部 機械航空宇宙工学科
%%%                   【非公式】学位論文 LaTeX テンプレート
%%%
%%%      <https://github.com/Yuki-MATSUKAWA/TUS-ME_thesis_template>
%%%
%%%                                  v1.0.0 Yuki MATSUKAWA 27 Dec. 2023
%%%
%%%%%%%%%%%%%%%%%%%%%%%%%%%%%%%%%%%%%%%%%%%%%%%%%%%%%%%%%%%%%%%%%%%%%%%

%%% 文書クラスの設定 %%%
\documentclass[
    paper=a4paper,      % A4 用紙サイズ
    report,             % report 相当の文書クラス
    fleqn,              % 数式を左寄せ
    fontsize=12pt,      % 欧文サイズ 12 pt
    jafontsize=12pt,    % 和文サイズ 12 pt
    head_space=33mm,    % 天の余白(柱とノンブルがあるので 20 mm よりも広い)
    foot_space=30mm,    % 地の余白(ノンブルが下の場合があるので 15 mm よりも広い)
    gutter=25mm,        % のどの余白
    fore-edge=10mm      % 小口の余白
    ]{jlreq}            % jlreq クラスを使用

%%% 学位論文設定ファイル %%%
\usepackage{settings}

%%% 行番号の表示 %%%
% 添削時には行番号を付けるとわかりやすい
% 提出時にはコメントアウトする
% \usepackage[mathlines,pagewise]{lineno}
% \linenumbers\relax


%%% ここから上を「プリアンブル」と言います.パッケージや独自の設定,マクロはプリアンブルや settings.styに書いてください.
%%% ここから下が論文の本体です.
\begin{document}

%%%%%%%%%%%%%%%%
%%%%% 表紙 %%%%%
%%%%%%%%%%%%%%%%

% 卒業・修了「年度」を入力
% \thesis{20**年度卒業論文}   % 卒業論文はこれ
\thesis{20**年度修士論文}   % 修士論文はこれ

% 学位論文題目
% ここには学位論文のタイトルを入れます.一文字でも間違えたら受理されません.
% タイトルが長くて改行するときは \\ を入れる.
\title{【非公式】機械航空宇宙工学科 学位論文テンプレート\\ ---\LaTeX で論文を書く際に必要な最低限の情報---}

% 卒業・修了「年」を入力
\date{20**年2月}

% 卒業論文の場合はこれ
% 大学名,学部名,学科名の間にスペースは不要
% \affiliation{東京理科大学創域理工学部機械航空宇宙工学科}

% 修士論文の場合はこれ
% 大学名,研究科名,専攻名の間にスペースは不要
\affiliation{東京理科大学大学院創域理工学研究科機械航空宇宙工学専攻}

% 研究室名を入力
\laboratory{〇〇研究室}

% 著者情報
\author{%
% 学籍番号を全角 7 桁で入力
75*****
\hskip2\zw% 学籍番号と氏名の間のスペース,消さない
% 姓と名の間は全角 1 文字スペース
姓姓 名名
} % 消さない

% 表紙の出力
\makecover

%%% 目次 %%%
\pagestyle{empty}
\def\thepage{}
\tableofcontents

%%% 記号表 %%%
\signary

% 英語のアルファベット順で並べる
\noindent
Alphabet\\
\begin{tblr}{
    colspec = {ll},         % 文字を左揃え
    colspec = {X[1]X[4]}    % 各列の幅が 1:4 になるように調整
}
    $d$     & Channel width [\si{m}] \\
    $L_j$   & Computational domain size in $j$-direction [\si{m}] \\
    $N_j$   & Number of grid points in $j$-direction \\
    $\Re$   & Reynolds number, $= ud/\nu$ \\
    $u$     & Velocity [\si{m/s}]
\end{tblr}

\hspace{30mm}

% ギリシャ語のアルファベット順で並べる
\noindent
Greek\\
\begin{tblr}{
    colspec = {ll},
    colspec = {X[1]X[4]}
}
    $\delta$            & Channel half width [\si{m}] \\
    $\epsilon_{ijk}$    & Levi--Civita symbol \\
    $\nu$               & Kinematic viscosity [\si{m^2/s}]
\end{tblr}

\hspace{30mm}

% 上付き添え字
\noindent
Superscripts\\
\begin{tblr}{
    colspec = {ll},
    colspec = {X[1]X[4]}
}
    $(\quad)^\ast$          & Normalized by outer variables, e.g., $\delta$ \\
    $(\quad)^+$             & Normalized by inner variables, e.g., $\nu/u_\tau$ (wall unit) \\
    $(\quad)^\prime$        & Fluctuation component \\
    $\overline{(\quad)}$    & Statistically averaged
\end{tblr}

\hspace{30mm}

% 下付き添え字
\noindent
Subscripts\\
\begin{tblr}{
    colspec = {ll},
    colspec = {X[1]X[4]}
}
    $(\quad)_\rms$          & Root mean square \\
    $(\quad)_\mathrm{w}$    & Wall \\
    $(\quad)_\tau$          & Wall unit
\end{tblr}





%%%%%%%%%%%%%%%%
%%%%% 本文 %%%%%
%%%%%%%%%%%%%%%%
\clearpage
\pagestyle{normal}
\setcounter{page}{0}
\pagenumbering{arabic}
% 上のコマンドは消さないで.
% 本文はこれ以降に記載する.

% LaTeX ソースは一つの tex ファイルに書くのではなく,章ごとの tex ファイルに分割して書きましょう.
% 分割したファイルを読み込むときは \include{xxx} または \input{xxx} を使います.
\chapter{はじめに}
\label{ch:introduction}

第\ref{ch:introduction}章では学位論文執筆の際の注意事項として,第\ref{sec:template}節でこのテンプレートの概要を,第\ref{sec:composition}節では GitHub リポジトリ内の各ファイルの説明を,第\ref{sec:abstract}節では卒論・修論要旨の \LaTeX テンプレートの紹介をします.
このテンプレートを使用する方は現在の \LaTeX 習熟度によらず必ず目を通してください.

\section{テンプレート概要}
\label{sec:template}

このファイルは東京理科大学創域理工学部機械航空宇宙工学科の卒業論文および同大学大学院創域理工学研究科機械航空宇宙工学専攻の修士論文を作成するにあたり,学科の論文執筆要件を満たした「非公式の」\LaTeX テンプレートです.
一連のファイルはテンプレート開発者\footnote{松川裕樹(東京理科大学大学院 創域理工学研究科 機械航空宇宙工学専攻 博士後期課程)\\ \quad Email: \texttt{7523701 _@_ ed.tus.ac.jp}}が管理している GitHub リポジトリ\footnote{\texttt{TUS-ME\_thesis\_template}: \textless\url{https://github.com/Yuki-MATSUKAWA/TUS-ME_thesis_template}\textgreater}から入手可能です.
また,私は熱流体系の研究室の所属ですが,所属研究室によらずこのテンプレートは使用可能です.
使用する際,特に許可を取る必要はありません.
ご自由にお使いください.

このテンプレートは研究室に配属されて初めて \LaTeX で文書を書くことになった学部 4 年生を対象に,環境構築から \verb|pdf| ファイルの生成,卒業論文執筆まで滞りなく行えるように作成しています.
そのため基本事項から説明をしていますが,表紙のタイトルにもある通り「必要最低限の情報」しか記載していません.
\LaTeX 入門書は既に良書がたくさんありますが,本当の初心者は知らなくてもいい情報や学位論文執筆だけを目指すうえでは不要な情報がたくさん書かれているため,困惑した読者も多いのではないかと思います.
このテンプレートには学位論文執筆をするうえで学生が欲しがるであろう情報のみを厳選し,その情報とこのテンプレートだけあれば学位論文を書き上げるくらいのことはできるようにしておきました.
そのため,\TeX/\LaTeX で一から文書を作成することを目指している方には情報が足りていないと思います.
さらに詳しい情報が欲しい人は書籍やインターネット上の情報を参考にしてください(第\ref{ch:information}章を参照).
また,この \verb|main.pdf| はモダンな \LaTeX である \LuaLaTeX で作成しているほか,\verb|jlreq| というドキュメントクラスや \verb|unicode-math| など最新の機能をふんだんに使用しています.
これからこのテンプレートを使い始めるという方はモダン \LaTeX を使えるようになっておきましょう.
しかし,学会の講演論文執筆の際はこれらの機能に対応していない場合もあるため,念のためレガシーな \LaTeX のコンパイル方法等についても説明をしています.
さらに,このテンプレートでは \LaTeX に関する説明はもちろんのこと,学生が論文を書くうえで躓きやすい箇所をまとめています.
特に表記に関して細かく記載しているので参考になる箇所は多いかと思います.

もしこのテンプレートに関してバグ等,使用上の問題が発生した際は GitHub の Issues にコメントしていただくかメールを送ってください.
このテンプレートの使用方法や \LaTeX に関する相談も遠慮なくどうぞ.
私が在学中であれば直接研究室に相談に来ていただいても構いません(ただし事前にメールでアポは取ってください).
可能な限り対応します.
ただし,このテンプレートを使用したことで生じた問題に関しては一切の責任を負いませんのでご了承ください.
また,少なくとも在学中はこのテンプレートのメンテナンス・更新は積極的に行うつもりです.
できるだけ最新版を使用するようにしてください.
学位取得後にテンプレートのメンテナンスをどうするかは私の進路によるところですが,そのタイミングで改めてここで説明します.

このテンプレートを使用される皆様が無事に学位論文を執筆し,卒業・修了されることを心の底から願っております.

\begin{flushright}
    \today \\
    松川裕樹
\end{flushright}

\clearpage
\section{リポジトリ内のファイル構成}
\label{sec:composition}

\begin{tcolorbox}[enhanced, title={\texttt{Yuki-MATSUKAWA/TUS-ME\_thesis\_template}}, drop fuzzy shadow]
    \begin{tabular}{ll}
        \verb|chapter/|     & 分割した \verb|tex| ファイルが入っているフォルダ \\
        \verb|figure/|      & 図が入っているフォルダ \\
        \verb|table/|       & 表の \verb|tex| ファイルが入っているフォルダ \\
        \verb|.gitignore|   & Git で管理しないファイル一覧 \\
        \verb|README.md|    & GitHub リポジトリの説明書 \\
        \verb|jsme.bst|     & 日本機械学会対応の \BibTeX スタイルファイル \\
        \verb|latexmkrc|    & 詳細は第\ref{ssec:latexmk}節を参照 \\
        \verb|main.pdf|     & \verb|main.tex| をコンパイルした \verb|pdf| ファイル \\
        \verb|main.tex|     & メインの文書ファイル \\
        \verb|mybib_en.bib| & 英語の参考文献リストファイル \\
        \verb|mybib_jp.bib| & 日本語の参考文献リストファイル \\
        \verb|settings.sty| & \verb|main.tex| で読み込むスタイルファイル
    \end{tabular}
\end{tcolorbox}

今皆さんが読んでいるこの \verb|pdf| ファイルは \verb|main.pdf| で,\verb|main.tex| を基に作成しています.
文書のレイアウト等細かい設定は全てスタイルファイル \verb|settings.sty| に書いています.
\verb|main.tex| 冒頭の \verb|\usepackage{settings}| で読み込んでいるので間違って消さないようにしてください.
\verb|main.tex| を適切なテキストエディター(第\ref{sec:editor}節を参照)で開いてもらうと,\verb|\include{chapter/xxx}| と書かれた文字列が複数目に入ってくると思います.
学位論文のような長い文書を一つの \verb|tex| ファイルに書き込むとわかりにくくなるので,\verb|chapter/| 以下のディレクトリに章(chapter)ごとに分割した \verb|tex| ファイルを置いておき,それを \verb|\include{}| コマンドで読み込むようにしています.
皆さんが学位論文を執筆する際にもこのように \verb|tex| ファイルを分割しておきましょう.


具体的なコンパイルの方法等については第\ref{ch:howtouse}章を参照してください.


\section{卒論・修論要旨}
\label{sec:abstract}

卒業論文・修士論文を提出する際は同時に要旨が必要です.
要旨についても \LaTeX テンプレートを作成したので,GitHub リポジトリ\footnote{\texttt{TUS-ME\_thesis\_abstract}: \textless\url{https://github.com/Yuki-MATSUKAWA/TUS-ME_thesis_abstract}\textgreater}からダウンロードしてください.
コンパイル方法はこのテンプレートと同様です.
要旨に関する詳細な説明はここでは省略しますが,\verb|README.md| にしっかりと目を通すようにしてください.



\chapter{\LaTeX の基本}
\label{ch:basic}

\lipsum[1]

\section{\LaTeX とは何か}
\label{sec:whatislatex}

\begin{equation}
    \Re_\rmin = \frac{u_\rmin h}{\nu}, \quad \Re_\rmout = \frac{u_\rmout h}{\nu}, \quad \Re = \frac{u_0 h}{\nu}
\end{equation}

\begin{align}
    \frac{\partial u_i}{\partial t} + u_j\frac{\partial u_i}{\partial x_j} &= - \frac{1}{\rho}\frac{\partial p}{\partial x_j} + \nu\frac{\partial^2 u_i}{\partial x_j \partial x_j} \\
    \pdv{u_i}{t} + u_j\pdv{u_i}{x_j} &= - \frac{1}{\rho}\pdv{p}{x_j} + \nu\pdv{u_i}{x_j}{x_j}
\end{align}

\section{\texttt{pdf}ファイルの生成}
\label{sec:makepdf}

\subsection{ターミナル上での操作}
\label{ssec:terminal}

\begin{tcolorbox}[enhanced, title=\pLaTeX, drop fuzzy shadow]
\begin{verbatim}
$ platex main (複数回)
$ dvipdfmx main
\end{verbatim}
\end{tcolorbox}


\begin{tcolorbox}[enhanced, title=\upLaTeX, drop fuzzy shadow]
\begin{verbatim}
$ uplatex main (複数回)
$ dvipdfmx main
\end{verbatim}
\end{tcolorbox}


\begin{tcolorbox}[enhanced, title=\pLaTeX$+$\pBibTeX, drop fuzzy shadow]
\begin{verbatim}
$ platex main
$ pbibtex main
$ platex main (複数回)
$ dvipdfmx main
\end{verbatim}
\end{tcolorbox}


\begin{tcolorbox}[enhanced, title=\upLaTeX$+$\upBibTeX, drop fuzzy shadow]
\begin{verbatim}
$ uplatex JSME-template1
$ upbibtex JSME-template1
$ uplatex JSME-template1 (複数回)
$ dvipdfmx JSME-template1
\end{verbatim}
\end{tcolorbox}

\begin{tcolorbox}[enhanced, title=\LuaLaTeX$+$\upBibTeX, drop fuzzy shadow]
\begin{verbatim}
$ lualatex JSME-template1
$ upbibtex JSME-template1
$ lualatex JSME-template1 (複数回)
\end{verbatim}
\end{tcolorbox}


\subsection{\texttt{latexmk}を使う方法}
\label{ssec:latexmk}

\lipsum[1-8]

\subsection{クラウド上で使う方法}
\label{ssec:cloud}


\lipsum[1-4]



\chapter{環境構築・操作方法}
\label{ch:howtouse}

第~\ref{ch:howtouse}~章では \TeX/\LaTeX 環境構築の方法と \verb|pdf| ファイルの生成までのプロセスを説明します.
第~\ref{sec:environment}~節では TeX Live のインストール方法について,第~\ref{sec:editor}~節では \TeX/\LaTeX 対応のテキストエディター,特に VS Code の場合について述べ,第~\ref{sec:makepdf}~節では \verb|pdf| ファイル生成までに必要なコマンドや \verb|latexmk| の使い方,クラウド上での \LaTeX の使用について述べます.

\section{環境構築}
\label{sec:environment}

\TeX/\LaTeX を使用する際は TeX Live というディストリビューションを ご自身の PC に入れましょう.
ウイルスバスターなどのウイルス対策ソフトが TeX Live のインストールを阻害するという問題が報告されているようです.
必ず阻害するわけではありませんが,一時的に動作を停止させておいてからインストールすることをオススメします.


\subsection{Windowsの場合}
\label{ssec:windows}

ここでは ISO イメージからのインストールとネットワークインストーラーからのインストールの二種類のインストール方法を説明します.
ISO イメージからインストールの方が問題は発生しにくいかもしれません.
一方でやってみてダメならもう一方で試してみてください.

\subsubsection*{ISO イメージからインストール}

\begin{enumerate}
    \item \textless\url{https://ftp.jaist.ac.jp/pub/CTAN/systems/texlive/Images/}\textgreater から \verb|texlive.iso| をダウンロード.
    \item ダブルクリックすると BD-ROM/DVD-ROM ドライブとしてマウントされる(「セキュリティの警告」が出た場合は「開く」を選択$\to$エクスプローラーで開く).
    \item 共通事項~\ref{enum:bat}へ.
\end{enumerate}


\subsubsection*{ネットワークインストーラーからのインストール}

\begin{enumerate}
    \item \textless\url{https://ftp.jaist.ac.jp/pub/CTAN/systems/texlive/tlnet/}\textgreater から \verb|install-tl.zip| をダウンロード.
    \item \verb|install-tl.zip| を展開.
    \item 共通事項~\ref{enum:bat}へ.
\end{enumerate}

\subsubsection*{共通事項}

\begin{enumerate}
    \setcounter{enumi}{3}
    \item \verb|install-tl-windows.bat| を実行(青い警告ウィンドウが出たら「詳細情報」$\to$「実行」).\label{enum:bat}
    \item TeX Liveインストーラが現れたら「TeXworksをインストール」のチェックを外してからインストール(もしTeXworksが欲しかったらインストールしてもよい).インストールは数時間かかることがあるので注意.
    \item インストールできたかどうかチェック.
    \begin{enumerate}
        \item \verb|Win|$+$\verb|R| でファイル名を \verb|cmd| と指定し \verb|cmd.exe|(コマンドプロンプトとも呼ぶ)を開く.
        \item \verb|tex -v| と入力し \verb|Enter|.
        \item バージョン情報が出てきたらインストール完了,出なかったら一度 Path を通してみる.
    \end{enumerate}
    \item 環境変数 Path の確認.
    \begin{enumerate}
        \item \verb|cmd.exe| を開く.\label{enum:path}
        \begin{enumerate}
            \item \verb|path| と入力し \verb|Enter|.
            \item \verb|C:\texlive\****\bin|(\verb|****| には TeX Live のバージョンにあてはまる年が入る)があれば完了.無ければ\ref{enum:system}へ.
        \end{enumerate}
        \item \verb|Win|$+$\verb|R| でファイル名を \verb|control| と入力し「コントロールパネル」を開く.\label{enum:system}
        \begin{enumerate}
            \item 「システム」$\to$「システムの詳細設定」$\to$「環境変数」の順に開く.
            \item 「システム環境変数」の「Path」をダブルクリック.
            \item \verb|C:\texlive\****\bin| があれば完了.無ければ「新規」で追加し,\ref{enum:path}へ.
        \end{enumerate}
    \end{enumerate}
\end{enumerate}

\subsection{macOSの場合}
\label{ssec:mac}



\section{使用するエディター}
\label{sec:editor}

\TeX/\LaTeX に対応しているテキストエディターは数多く存在しますが,ここでは Microsoft が開発している Visual Studio Code(VS Code)を紹介します.
開発元は Microsoft ですが,Windows だけでなく macOS や Linux でも使用可能です.
また,VS Code には豊富な拡張機能が存在しているほか,Git との連携も非常に簡単なため近年非常に人気の高いエディターです.
VS Code の詳細な使用方法はここでは割愛しますが,最低限の拡張機能として \href{https://marketplace.visualstudio.com/items?itemName=James-Yu.latex-workshop}{LaTeX Workshop} を入れておくとよいでしょう.
取り扱う画像ファイルが多くなってきた場合は \href{https://marketplace.visualstudio.com/items?itemName=kisstkondoros.vscode-gutter-preview}{Image preview} があると便利です.
また,VS Code の設定ファイル(\verb|settings.json|)でさまざまな設定を書き加えることができます.
一例として私が使っている \verb|settings.json|\footnote{\textless\url{https://gist.github.com/Yuki-MATSUKAWA/465ecd0ebcbd157e48ac1e3619c9a08c}\textgreater} を紹介しておきます.
\LaTeX 以外の設定も含まれているので設定の取捨選択は読者の皆さんにお任せします.

\section{\texttt{pdf}ファイルの生成}
\label{sec:makepdf}

ここでは実際に \verb|pdf| ファイル(このテンプレートでは \verb|main.pdf|)を生成する過程を説明します.
これまでにある程度 \LaTeX を使った経験のある方は必要な箇所だけ読めばいい(全部わかっていれば読む必要は無い)と思います.
と言っても,\LaTeX 初心者も全部を読む必要は無く,「一旦このテンプレートで学位論文を書き上げたい」ということを考えている人は第~\ref{ssec:terminal}~節の「\LuaLaTeX の場合(このテンプレートはこちら)」と第~\ref{ssec:latexmk}~節を読めば大丈夫です.

\subsection{ターミナル上での操作}
\label{ssec:terminal}

第~\ref{ssec:latexmk}~節の \verb|latexmk| を使用すればターミナル上での操作は非常に簡単になりますが,何か問題が発生した際にデバッグをすることを考えるとターミナル上での操作も覚えておく必要があります.
実際に \verb|pdf| ファイルを生成するときは \verb|latexmk| を使用すればいいのですが,まずはどのようなプロセスで実行されているのかを把握しておきましょう.

\subsubsection*{\LuaLaTeX の場合(このテンプレートはこちら)}

この \LaTeX テンプレートは \LuaLaTeX での執筆を前提とし,参考文献は \upBibTeX で読み込むようにしています.
\LuaLaTeX は速度がやや遅いものの,高機能で Unicode に対応しているため近年人気が出てきているモダンな \LaTeX です.
使い方の詳細は下記のようになります.

\begin{tcolorbox}[enhanced, title=\LuaLaTeX$+$\upBibTeX, drop fuzzy shadow]
\begin{verbatim}
$ lualatex main
$ upbibtex main
$ lualatex main (複数回)
\end{verbatim}
\end{tcolorbox}

まずは主要な \LaTeX ソースコードの \verb|main.tex| を \LuaLaTeX で読み込むために \verb|lualatex main| とターミナルに入力します.
\verb|$| は入力しないでください.
拡張子の \verb|.tex| は省略可能です.
次に,参考文献を読み込むために \verb|upbibtex main| とターミナルに入力します.
\BibTeX を使わない処理をしているときはこの操作は不要です.
これだけだとまだ \LaTeX を使う大きなメリットである相互参照の機能を使えていません.
\LaTeX で相互参照を有効にするには複数回のコンパイルが必要です.
相互参照に失敗した場合やコンパイル回数が足りていない場合は参照箇所が ? や ?? のように表示されるはずです.
そのため,\upBibTeX を読み込んだ後に ? や ?? が消えるまで複数回コンパイルしましょう.
これで \verb|main.pdf| を作成できました.

\subsubsection*{レガシー \LaTeX の場合}

モダン \LaTeX とレガシー \LaTeX の最大の違いは,\verb|pdf| ファイルを直接生成できるか否かです.
\pLaTeX や \upLaTeX のようなレガシー \LaTeX は一度 \verb|dvi| ファイルという中間ファイルを生成し,その後 \verb|dvi| ファイルを \verb|pdf| 等の適切なファイル形式に変換する作業が必要です(\verb|dvipdfmx|).
これからの時代はどんどんモダン \LaTeX に置き換えられていくと思いますが,まだ対応していない学会・論文テンプレートも多く存在しているのでここで紹介しておきます.
また,\pdfLaTeX は本来レガシー \LaTeX ですが,例外的に直接 \verb|pdf| ファイルを生成でき,国際雑誌論文テンプレートではよく使用されています.
ただし,\pdfLaTeX は日本語に対応していないため,日本語を使用したい人は \LuaLaTeX を使うようにしましょう.
どうしても \pdfLaTeX で日本語を使用したい(国際雑誌論文執筆の下書き等)場合は第~\ref{ssec:pdflatex_jp}~節を参照してください.

\pLaTeX は日本語に対応した \LaTeX として長年愛用されてきましたが,今は \LuaLaTeX などに置き換えられてきています.
皆さんは使わないようにしましょう.
使い方は下記の通り.
\LuaLaTeX の項目と同様,\BibTeX を使わない場合はそこのコマンドを省略してください.

\begin{tcolorbox}[enhanced, title=\pLaTeX$+$\pBibTeX, drop fuzzy shadow]
\begin{verbatim}
$ platex main
$ pbibtex main
$ platex main (複数回)
$ dvipdfmx main
\end{verbatim}
または
\begin{verbatim}
$ ptex2pdf -l main
$ pbibtex main
$ ptex2pdf -l main (複数回)
\end{verbatim}
\end{tcolorbox}

上記コマンドの \verb|ptex2pdf -l main| は \verb|platex main| と \verb|dvipdfmx main| を続けて実行するコマンドです.

次に \upLaTeX について説明します.
これは \pLaTeX を Unicode に対応させたものとなっており,現在でも広く使われています.
そのため,このテンプレートを使用することだけを考える際は不要な情報ですが,念のため載せておきます.
\upLaTeX では \upBibTeX が使えますが先程と同様,不要な場合は省略してください.

\begin{tcolorbox}[enhanced, title=\upLaTeX$+$\upBibTeX, drop fuzzy shadow]
\begin{verbatim}
$ uplatex main
$ upbibtex main
$ uplatex main (複数回)
$ dvipdfmx main
\end{verbatim}
または
\begin{verbatim}
$ ptex2pdf -l -u main
$ pbibtex main
$ ptex2pdf -l -u main (複数回)
\end{verbatim}
\end{tcolorbox}

\pLaTeX はレガシー \LaTeX ですが例外的に直接 \verb|pdf| を出力できます(\verb|dvipdfmx| が不要).
日本語には対応していませんが,国際雑誌論文では広く使用されています.
どうしても \pdfLaTeX で日本語を使用したい場合は次の第~\ref{ssec:pdflatex_jp}~節を参照.
使い方は下記の通り.

\begin{tcolorbox}[enhanced, title=\pdfLaTeX$+$\upBibTeX, drop fuzzy shadow]
\begin{verbatim}
$ pdflatex main
$ upbibtex main
$ pdflatex main (複数回)
\end{verbatim}
\end{tcolorbox}


\subsection{\pdfLaTeX で日本語を使用する場合}
\label{ssec:pdflatex_jp}

国際雑誌論文等のコンパイルは \pdfLaTeX が想定されていることがあります.
\pdfLaTeX はレガシー \LaTeX でありながらも直接 \verb|pdf| ファイルを生成できることから海外では広く使用されていますが,残念ながら日本語に対応していません.
しかし,英語論文の下書きとして日本語を使いたい場合があると思います.
その際に,見た目が少し悪くなるものの \pdfLaTeX で日本語を使用する方法が一応あるのでここで紹介しておきます.

\begin{tcolorbox}[enhanced, title=文書全体で日本語を使用, drop fuzzy shadow]
\begin{verbatim}
\usepackage[whole]{bxcjkjatype}
\end{verbatim}
\end{tcolorbox}

まず,\LaTeX 文書全体で日本語を使用したい場合は上記のコマンドをプリアンブルに書きます.
これで文書全体で日本語の使用が可能になります.
ただし,前述の通り見た目が悪くなるので下書き用(後で英語に変更する用)として使用してください.

\begin{tcolorbox}[enhanced, title=文書の一部分で日本語を使用, drop fuzzy shadow]
\begin{verbatim}
プリアンブルに記載
\usepackage{CJKutf8}

本文中に記載
\begin{CJK}{UTF8}{ipxm}
日本語
\end{CJK}
\end{verbatim}
\end{tcolorbox}

次に,文書全体ではなく一部分でのみ日本語を使用したい場合のコマンドは上記のようになっています.
まず,\verb|\usepackage{CJKutf8}| というパッケージを読み込むことで日本語を使用できるようにします.
厳密には日本語だけでなく,中国語(\textbf{C}hinese),日本語(\textbf{J}apanese),韓国語(\textbf{K}orean)の組版規則に対応させるためのパッケージとなります.
次に本文中の日本語を使いたい箇所を \verb|\begin{CJK}{UTF8}{ipxm}| と \verb|\end{CJK}| で囲ってあげればそこでは日本語を使えるようになります.
米国物理学協会(American Institute of Physics, AIP)が発行している雑誌論文(Physics of Fluids など)は著者の氏名で英語表記以外に漢字等の表記を併記することが可能になっています.
このようなときにこのコマンドを使ってあげるとよいでしょう.
また,日本語を使う箇所がもう少し長い場合はプリアンブルで \verb|\newcommand*{\Ja}[1]{\begin{CJK}{UTF8}{ipxm}#1\end{CJK}}| のようにコマンドを作ってあげてもいいかもしれません.

\subsection{\texttt{latexmk}を使う方法}
\label{ssec:latexmk}




\subsection{クラウド上で使う方法}
\label{ssec:cloud}

第~\ref{sec:environment}~節で環境構築の方法を述べました.
私の本音としては \LaTeX の全ユーザーが自身の PC にローカルの \LaTeX 環境を整えてほしいのですが,環境構築に手間がかかる不便さもあるため,ここでは TeX Live 等のインストールをせずにクラウド上での \LaTeX 環境構築方法を説明します.

クラウド上で \LaTeX を使用できるツールとしては Cloud LaTeX や Overleaf といったものが有名です.

Overleaf は複数のユーザーによる(同時)共同執筆も可能となっており,Overleaf 自体が Git と同様の役割を担っているため大変便利です.
もちろん Git/GitHub との連携も可能となっています.





\chapter{\BibTeX による参考文献一覧の出力}
\label{ch:biblist}


\section{参考文献記載時の一般的な注意事項}
\label{sec:bibcaution}


\section{\BibTeX とは何か}
\label{sec:bibtex}


\section{\texttt{jsme.bst}}
\label{sec:jsme-bst}




\chapter{先生や先輩に添削してもらうときの注意点}
\label{ch:check}

\lipsum[1-10]

\section{\texttt{latexdiff}}
\label{sec:latexdiff}

\lipsum[1-8]


\section{\texttt{latexdiff-vc}}
\label{sec:latexdiff-vc}



\chapter{図表の配置}
\label{ch:figure_table}



\section{図の配置}
\label{sec:figure}


\begin{figure}
    \centering
    \includegraphics[width=0.5\textwidth]{figure/tiger1.pdf}
    \caption{一枚の図.}
    \label{fig:example}
\end{figure}


\begin{figure}
    \centering
    \begin{subfigure}{0.45\textwidth}
        \centering
        \includegraphics[width=\textwidth]{example-image-a}
        \caption{左の図.}
        \label{subfig:example_a}
    \end{subfigure}
    \hfill % ここで空白を入れると図が適切に配置される
    \begin{subfigure}{0.45\textwidth}
        \centering
        \includegraphics[width=\textwidth]{example-image-b}
        \caption{右の図.}
        \label{subfig:example_b}
    \end{subfigure}
    \caption{左右の図.}
    \label{fig:example2}
\end{figure}




\section{表の配置}
\label{sec:table}






\chapter{さらに詳しい情報が欲しい人は}
\label{ch:information}


\section{書籍}
\label{sec:book}

\section{インターネット上の情報}
\label{sec:internet}





%%% 謝辞 %%%
%%%%%%%%%%%%%%%%
%%%%% 謝辞 %%%%%
%%%%%%%%%%%%%%%%
\acknowledge

\lipsum[1-20]



%%% 文献 %%%
\biblist
% 使用する bst ファイル
\bibliographystyle{jsme}
% 読み込む bib ファイル
\bibliography{
    mybib_en.bib,
    mybib_jp.bib
}

%%% 付録 %%%
%%%%%%%%%%%%%%%%
%%%%% 付録 %%%%%
%%%%%%%%%%%%%%%%
\appendix
\label{ch:app}
\pagestyle{appendix}

\chapter{修士課程における研究成果}
\label{ch:app_master}

\lipsum[1-8]

\chapter{スーパーコンピューターごとの性能比較}
\label{ch:app_sx}

\lipsum[1-2]

\section{スパコンXXX}
\label{sec:app_xxx}

\lipsum[1-4]

\section{スパコンYYY}
\label{sec:app_yyy}

\lipsum[1-4]


%%%%%%%%%%%%%%%%%%%%%%%%%%%%%%
%%% 文章を書けるのはここまで %%%
%%%%%%%%%%%%%%%%%%%%%%%%%%%%%%
\end{document}
